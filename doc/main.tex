\documentclass{article}
\usepackage{luatextra}
\usepackage{polyglossia}
\usepackage{ulem}
\usepackage{framed}
\usepackage{color}
\usepackage{geometry}
\usepackage{amsmath}
\usepackage{unicode-math}
\usepackage[hidelinks]{hyperref}
\usepackage{latexsym}
\usepackage{pdflscape}
\usepackage{pdfpages}
\usepackage{enumitem}
\usepackage{titlesec}
\usepackage{lastpage}
\usepackage{fancyhdr}
\usepackage{titlesec}
\usepackage{listings}

\usepackage{ifluatex}
\ifluatex
  \usepackage{pdftexcmds}
  \makeatletter
  \let\pdfstrcmp\pdf@strcmp
  \let\pdffilemoddate\pdf@filemoddate
  \makeatother
\fi
\usepackage{svg}

\setmathfont{xits-math.otf}

\setmainlanguage{french}
\selectlanguage{french}
%%\setmainfont{Latin Modern Roman}
\setmainfont{Roboto}

\geometry{margin={1in,1in}}

\setlist{nosep} %% No space between lists' items

\pagestyle{fancy}
\fancyhead[R]{}

%% <current page>/<total pages> footer
\cfoot{\thepage/\pageref{LastPage}}

\newcommand\image[2]{
\directlua{
local image = img.scan({filename = "#1"})

image.height = image.height * #2
image.width  = image.width  * #2

node.write(img.node(image))
}
}


%%%%%%%%%%%%%%%%%%%%%%%%%%%%%%%%%%%%%%%%%%
%% \subsubsubsection command definition %%
%%%%%%%%%%%%%%%%%%%%%%%%%%%%%%%%%%%%%%%%%%


\titleclass{\subsubsubsection}{straight}[\subsection]

\newcounter{subsubsubsection}[subsubsection]
\renewcommand\thesubsubsubsection{\thesubsubsection.\arabic{subsubsubsection}}
\renewcommand\theparagraph{\thesubsubsubsection.\arabic{paragraph}} % optional; useful if paragraphs are to be numbered

\titleformat{\subsubsubsection}
  {\normalfont\normalsize\bfseries}{\thesubsubsubsection}{1em}{}
\titlespacing*{\subsubsubsection}
{0pt}{3.25ex plus 1ex minus .2ex}{1.5ex plus .2ex}

\makeatletter
\renewcommand\paragraph{\@startsection{paragraph}{5}{\z@}%
  {3.25ex \@plus1ex \@minus.2ex}%
  {-1em}%
  {\normalfont\normalsize\bfseries}}
\renewcommand\subparagraph{\@startsection{subparagraph}{6}{\parindent}%
  {3.25ex \@plus1ex \@minus .2ex}%
  {-1em}%
  {\normalfont\normalsize\bfseries}}
\def\toclevel@subsubsubsection{4}
\def\toclevel@paragraph{5}
\def\toclevel@paragraph{6}
\def\l@subsubsubsection{\@dottedtocline{4}{7em}{4em}}
\def\l@paragraph{\@dottedtocline{5}{10em}{5em}}
\def\l@subparagraph{\@dottedtocline{6}{14em}{6em}}
\makeatother

\setcounter{secnumdepth}{4}
\setcounter{tocdepth}{4}

%%%%%%%%%%%%%%%%%%%%%%%%%%%%%%%%%%%%%%%%%%
%% end \subsubsubsection definition     %%
%%%%%%%%%%%%%%%%%%%%%%%%%%%%%%%%%%%%%%%%%%


\titlespacing*{\section}{0pt}{0.7\baselineskip}{0.7\baselineskip}

\title{Projet Principe d'éxécution des programmes}
%\subtitle{Meow}
\author{NassimBounouas, piernov}
\date{\today}

\begin{document}
\lstset{xleftmargin=.25in}

\maketitle
\textcolor{blue}{\url{https://github.com/NassimBounouas/Stage_SI3_PeP}}
\tableofcontents

\newpage

\section{Présentation du projet}
\subsection{Le microprocesseur ARM Cortex-M0}
	Le microprocesseur ARM Cortex-M0 

\subsection{Assembleur}

\subsubsection{Syntaxe}
Syntaxe UAL\\
S: màj des drapeaux\\
<c>: condition\\
Rd: registre destination\\
<imm?>: immédiat\\
SP: registre de pointeur de pile en mémoire\\
opcode: code de l'instruction, peut occuper jusqu'à la taille indiquée\\

\section{Jeu d'instructions (Instruction Set Architecture)}

\subsection{Instructions à implémenter}

\textbf{Binaire:}\\

\begin{tabular}{| c c c c c c c c c c c c c c c c |}
\hline
15 & 14 & 13 & 12 & 11 & 10 & \multicolumn{1}{|c}{9} & 8 & 7 & 6 & 5 & 4 & 3 & 2 & 1 & 0 \\
\hline
\multicolumn{6}{|c}{opcode} & \multicolumn{10}{|c|}{} \\
\hline
\end{tabular}


\subsubsection{Shift, add, sub, mov}

\textbf{Binaire:}\\

\begin{tabular}{| c c c c c c c c c c c c c c c c |}
\hline
15 & 14 & \multicolumn{1}{|c}{13} & 12 & 11 & 10 & 9 & \multicolumn{1}{|c}{8} & 7 & 6 & 5 & 4 & 3 & 2 & 1 & 0 \\
\hline
0 & 0 & \multicolumn{5}{|c}{opcode} & \multicolumn{9}{|c|}{} \\
\hline
\end{tabular}


\subsubsubsection{LSL (immediate): Logical Shift Left (p. 298)}

\textbf{Description: }

Décale le contenu du registre \texttt{Rm} vers la gauche d'un nombre de bits donné par l'immédiat \texttt{imm5}, écrit le résultat dans le registre \texttt{Rd}.\\
Des zéros sont insérés à droite.\\
Les drapeaux sont mis à jour.\\

\textbf{Assembleur:} T1

\begin{lstlisting}
LSLS <Rd>,<Rm>,#<imm5>
\end{lstlisting}

\textbf{Binaire:}\\

\begin{tabular}{| c c c c c c c c c c c c c c c c |}
\hline
15 & 14 & 13 & \multicolumn{1}{|c}{12} & 11 & \multicolumn{1}{|c}{10} & 9 & 8 & 7 & 6 & \multicolumn{1}{|c}{5} & 4 & 3 & \multicolumn{1}{|c}{2} & 1 & 0 \\
\hline
0 & 0 & 0 & \multicolumn{1}{|c}{0} & 0 & \multicolumn{5}{|c|}{imm5} & \multicolumn{3}{|c|}{Rm} & \multicolumn{3}{|c|}{Rd} \\
\hline
\end{tabular}

\subsubsubsection{LSR (immediate): Logical Shift Right (p. 302)}

\textbf{Description: }

Décale le contenu du registre \texttt{Rm} vers la droite d'un nombre de bits donné par l'immédiat \texttt{imm5}, écrit le résultat dans le registre \texttt{Rd}.\\
Des zéros sont insérés à gauche.\\
Les drapeaux sont mis à jour.\\

\textbf{Assembleur:} T1

\begin{lstlisting}
LSRS <Rd>,<Rm>,#<imm5>
\end{lstlisting}

\textbf{Binaire:}\\

\begin{tabular}{| c c c c c c c c c c c c c c c c |}
\hline
15 & 14 & 13 & \multicolumn{1}{|c}{12} & 11 & \multicolumn{1}{|c}{10} & 9 & 8 & 7 & 6 & \multicolumn{1}{|c}{5} & 4 & 3 & \multicolumn{1}{|c}{2} & 1 & 0 \\
\hline   
0 & 0 & 0 & \multicolumn{1}{|c}{0} & 1 & \multicolumn{5}{|c|}{imm5} & \multicolumn{3}{|c|}{Rm} & \multicolumn{3}{|c|}{Rd} \\
\hline
\end{tabular}


\subsubsubsection{ASR (immediate): Arithmetic Shift Right (p. 203)}

\textbf{Description: }

Décale le contenu du registre \texttt{Rm} vers la droite d'un nombre de bits donné par l'immédiat \texttt{imm5}, écrit le résultat dans le registre \texttt{Rd}.\\
Le bit de signe de \texttt{Rm} est ré-inséré à gauche.\\
Les drapeaux sont mis à jour.\\

\textbf{Assembleur:} T1

\begin{lstlisting}
ASRS <Rd>,<Rm>,#<imm5>
\end{lstlisting}

\textbf{Binaire:}\\

\begin{tabular}{| c c c c c c c c c c c c c c c c |}
\hline
15 & 14 & 13 & \multicolumn{1}{|c}{12} & 11 & \multicolumn{1}{|c}{10} & 9 & 8 & 7 & 6 & \multicolumn{1}{|c}{5} & 4 & 3 & \multicolumn{1}{|c}{2} & 1 & 0 \\
\hline   
0 & 0 & 0 & \multicolumn{1}{|c}{1} & 0 & \multicolumn{5}{|c|}{imm5} & \multicolumn{3}{|c|}{Rm} & \multicolumn{3}{|c|}{Rd} \\
\hline
\end{tabular}


\subsubsubsection{ADD (register): Add register (p. 191)}

\textbf{Description: }

Ajoute le contenu du registre \texttt{Rn} au contenu du registre \texttt{Rm}, écrit le résultat dans le registre \texttt{Rd}.\\
Les drapeaux sont mis à jour.\\

\textbf{Assembleur:} T1

\begin{lstlisting}
ADDS <Rd>,<Rn>,<Rm>
\end{lstlisting}

\textbf{Binaire:}\\

\begin{tabular}{| c c c c c c c c c c c c c c c c |}
\hline
15 & 14 & 13 & \multicolumn{1}{|c}{12} & 11 & \multicolumn{1}{|c}{10} & \multicolumn{1}{|c}{9} & \multicolumn{1}{|c}{8} & 7 & 6 & \multicolumn{1}{|c}{5} & 4 & 3 & \multicolumn{1}{|c}{2} & 1 & 0 \\
\hline   
0 & 0 & 0 & \multicolumn{1}{|c}{1} & 1 &  \multicolumn{1}{|c}{0} & \multicolumn{1}{|c}{0} & \multicolumn{3}{|c|}{Rm} & \multicolumn{3}{|c|}{Rn} & \multicolumn{3}{|c|}{Rd} \\
\hline
\end{tabular}


\subsubsubsection{SUB (register): Substract register (p. 450)}

\textbf{Description: }
Soustrait le contenu du registre \texttt{Rn} au contenu du registre \texttt{Rm}, écrit le résultat dans le registre \texttt{Rd}.\\
Les drapeaux sont mis à jour.\\

\textbf{Assembleur:} T1

\begin{lstlisting}
SUBS <Rd>,<Rn>,<Rm>
\end{lstlisting}

\textbf{Binaire:}\\

\begin{tabular}{| c c c c c c c c c c c c c c c c |}
\hline
15 & 14 & 13 & \multicolumn{1}{|c}{12} & 11 & \multicolumn{1}{|c}{10} & \multicolumn{1}{|c}{9} & \multicolumn{1}{|c}{8} & 7 & 6 & \multicolumn{1}{|c}{5} & 4 & 3 & \multicolumn{1}{|c}{2} & 1 & 0 \\
\hline   
0 & 0 & 0 & \multicolumn{1}{|c}{1} & 1 &  \multicolumn{1}{|c}{0} & \multicolumn{1}{|c}{1} & \multicolumn{3}{|c|}{Rm} & \multicolumn{3}{|c|}{Rn} & \multicolumn{3}{|c|}{Rd} \\
\hline
\end{tabular}

\subsubsubsection{MOV (immediate): Move (p. 312)}

\textbf{Description: }

Écrit l'immédiat \texttt{imm8} dans le registre \texttt{Rd}.\\
Les drapeaux sont mis à jour.\\

\textbf{Assembleur:} T1

\begin{lstlisting}
MOVS <Rd>,#<imm8>
\end{lstlisting}

\textbf{Binaire:}\\

\begin{tabular}{| c c c c c c c c c c c c c c c c |}
\hline
15 & 14 & 13 & \multicolumn{1}{|c}{12} & 11 & \multicolumn{1}{|c}{10} & 9 & 8 & \multicolumn{1}{|c}{7} & 6 & 5 & 4 & 3 & 2 & 1 & 0 \\
\hline
0 & 0 & 1 & \multicolumn{1}{|c}{0} & 0 & \multicolumn{3}{|c|}{Rd} & \multicolumn{8}{|c|}{imm8} \\
\hline
\end{tabular}


\subsubsection{Data processing}

\textbf{Binaire:}\\

\begin{tabular}{| c c c c c c c c c c c c c c c c |}
\hline
15 & 14 & 13 & 12 & 11 & 10 & \multicolumn{1}{|c}{9} & 8 & 7 & 6 & \multicolumn{1}{|c}{5} & 4 & 3 & 2 & 1 & 0 \\
\hline
0 & 1 & 0 & 0 & 0 & 0 & \multicolumn{4}{|c}{opcode} & \multicolumn{6}{|c|}{} \\
\hline
\end{tabular}

\subsubsubsection{AND (register): Bitwise AND (p. 201)}

\textbf{Description: }

Effectue un ET binaire entre le contenu du registre \texttt{Rdn} et le contenu du registre \texttt{Rm}, écrit le résultat dans le registre \texttt{Rdn}.\\
Les drapeaux sont mis à jour.\\

\textbf{Assembleur:} T1

\begin{lstlisting}
ANDS <Rdn>,<Rm>
\end{lstlisting}

\textbf{Binaire:}\\

\begin{tabular}{| c c c c c c c c c c c c c c c c |}
\hline
15 & 14 & 13 & 12 & 11 & 10 & \multicolumn{1}{|c}{9} & 8 & 7 & 6 & \multicolumn{1}{|c}{5} & 4 & 3 & \multicolumn{1}{|c}{2} & 1 & 0 \\
\hline
0 & 1 & 0 & 0 & 0 & 0 & \multicolumn{1}{|c}{0} & 0 & 0 & 0 & \multicolumn{3}{|c}{Rm} & \multicolumn{3}{|c|}{Rdn} \\
\hline
\end{tabular}


\subsubsubsection{EOR (register): Exclusive OR (p. 239)}

\textbf{Description: }

Effectue un OU exclusif binaire entre le contenu du registre \texttt{Rdn} et le contenu du registre \texttt{Rm}, écrit le résultat dans le registre \texttt{Rdn}.\\
Les drapeaux sont mis à jour.\\

\textbf{Assembleur:} T1

\begin{lstlisting}
EORS <Rdn>,<Rm>
\end{lstlisting}

\textbf{Binaire:}\\

\begin{tabular}{| c c c c c c c c c c c c c c c c |}
\hline
15 & 14 & 13 & 12 & 11 & 10 & \multicolumn{1}{|c}{9} & 8 & 7 & 6 & \multicolumn{1}{|c}{5} & 4 & 3 & \multicolumn{1}{|c}{2} & 1 & 0 \\
\hline
0 & 1 & 0 & 0 & 0 & 0 & \multicolumn{1}{|c}{0} & 0 & 0 & 1 & \multicolumn{3}{|c}{Rm} & \multicolumn{3}{|c|}{Rdn} \\
\hline
\end{tabular}


\subsubsubsection{LSL (register): Logical Shift Left (p. 300)}

\textbf{Description: }

Décale le contenu du registre \texttt{Rdn} vers la gauche d'un nombre de bits donné par l'octet inférieur du registre \texttt{Rm}, écrit le résultat dans le registre \texttt{Rdn}.\\
Des zéros sont insérés à droite.\\
Les drapeaux sont mis à jour.\\

\textbf{Assembleur:} T1

\begin{lstlisting}
LSLS <Rdn>,<Rm>
\end{lstlisting}

\textbf{Binaire:}\\

\begin{tabular}{| c c c c c c c c c c c c c c c c |}
\hline
15 & 14 & 13 & 12 & 11 & 10 & \multicolumn{1}{|c}{9} & 8 & 7 & 6 & \multicolumn{1}{|c}{5} & 4 & 3 & \multicolumn{1}{|c}{2} & 1 & 0 \\
\hline
0 & 1 & 0 & 0 & 0 & 0 & \multicolumn{1}{|c}{0} & 0 & 1 & 0 & \multicolumn{3}{|c}{Rm} & \multicolumn{3}{|c|}{Rdn} \\
\hline
\end{tabular}



\subsubsubsection{LSR (register): Logical Shift Right (p. 304)}

\textbf{Description: }

Décale le contenu du registre \texttt{Rdn} vers la droite d'un nombre de bits donné par l'octet inférieur du registre \texttt{Rm}, écrit le résultat dans le registre \texttt{Rdn}.\\
Des zéros sont insérés à gauche.\\
Les drapeaux sont mis à jour.\\

\textbf{Assembleur:} T1

\begin{lstlisting}
LSRS <Rdn>,<Rm>
\end{lstlisting}

\textbf{Binaire:}\\

\begin{tabular}{| c c c c c c c c c c c c c c c c |}
\hline
15 & 14 & 13 & 12 & 11 & 10 & \multicolumn{1}{|c}{9} & 8 & 7 & 6 & \multicolumn{1}{|c}{5} & 4 & 3 & \multicolumn{1}{|c}{2} & 1 & 0 \\
\hline
0 & 1 & 0 & 0 & 0 & 0 & \multicolumn{1}{|c}{0} & 0 & 1 & 1 & \multicolumn{3}{|c}{Rm} & \multicolumn{3}{|c|}{Rdn} \\
\hline
\end{tabular}


\subsubsubsection{ASR (register): Arithmetic Shift Right (p. 205)}

\textbf{Description: }

Décale le contenu du registre \texttt{Rdn} vers la droite d'un nombre de bits donné par l'octet inférieur du registre \texttt{Rm}, écrit le résultat dans le registre \texttt{Rdn}.\\
Le bit de signe de \texttt{Rdn} est ré-inséré à gauche.\\
Les drapeaux sont mis à jour.\\

\textbf{Assembleur:} T1

\begin{lstlisting}
ASRS <Rdn>,<Rm>
\end{lstlisting}

\textbf{Binaire:}\\

\begin{tabular}{| c c c c c c c c c c c c c c c c |}
\hline
15 & 14 & 13 & 12 & 11 & 10 & \multicolumn{1}{|c}{9} & 8 & 7 & 6 & \multicolumn{1}{|c}{5} & 4 & 3 & \multicolumn{1}{|c}{2} & 1 & 0 \\
\hline
0 & 1 & 0 & 0 & 0 & 0 & \multicolumn{1}{|c}{0} & 1 & 0 & 0 & \multicolumn{3}{|c}{Rm} & \multicolumn{3}{|c|}{Rdn} \\
\hline
\end{tabular}


\subsubsubsection{ADC (register): Add with Carry (p. 187)}

\textbf{Description: }

Ajoute le contenu du registre \texttt{Rm} et le drapeau de retenu au contenu du registre \texttt{Rdn}, écrit le résultat dans le registre \texttt{Rdn}.\\
Les drapeaux sont mis à jour.\\

\textbf{Assembleur:} T1

\begin{lstlisting}
ADCS <Rdn>,<Rm>
\end{lstlisting}

\textbf{Binaire:}\\

\begin{tabular}{| c c c c c c c c c c c c c c c c |}
\hline
15 & 14 & 13 & 12 & 11 & 10 & \multicolumn{1}{|c}{9} & 8 & 7 & 6 & \multicolumn{1}{|c}{5} & 4 & 3 & \multicolumn{1}{|c}{2} & 1 & 0 \\
\hline
0 & 1 & 0 & 0 & 0 & 0 & \multicolumn{1}{|c}{0} & 1 & 0 & 1 & \multicolumn{3}{|c}{Rm} & \multicolumn{3}{|c|}{Rdn} \\
\hline
\end{tabular}

\subsubsubsection{SBC (register): Substract with Carry (p. 380)}

\textbf{Description: }

Soustrait le contenu du registre \texttt{Rm} et le complément du drapeau de retenu au contenu du registre \texttt{Rdn}, écrit le résultat dans le registre \texttt{Rdn}.\\
Les drapeaux sont mis à jour.\\

\textbf{Assembleur:} T1

\begin{lstlisting}
SBCS <Rdn>,<Rm>
\end{lstlisting}

\textbf{Binaire:}\\

\begin{tabular}{| c c c c c c c c c c c c c c c c |}
\hline
15 & 14 & 13 & 12 & 11 & 10 & \multicolumn{1}{|c}{9} & 8 & 7 & 6 & \multicolumn{1}{|c}{5} & 4 & 3 & \multicolumn{1}{|c}{2} & 1 & 0 \\
\hline
0 & 1 & 0 & 0 & 0 & 0 & \multicolumn{1}{|c}{0} & 1 & 1 & 0 & \multicolumn{3}{|c}{Rm} & \multicolumn{3}{|c|}{Rdn} \\
\hline
\end{tabular}



\subsubsubsection{ROR (register): Rotate Right (p. 368)}

\textbf{Description: }

Pivote le contenu du registre \texttt{Rdn} vers la droite d'un nombre de bits donné par l'octet inférieur du registre \texttt{Rm}, écrit le résultat dans le registre \texttt{Rdn}.\\
Les bits de \texttt{Rdn} sortant à droite sont ré-insérés à gauche.\\
Les drapeaux sont mis à jour.\\

\textbf{Assembleur:} T1

\begin{lstlisting}
RORS <Rdn>,<Rm>
\end{lstlisting}

\textbf{Binaire:}\\

\begin{tabular}{| c c c c c c c c c c c c c c c c |}
\hline
15 & 14 & 13 & 12 & 11 & 10 & \multicolumn{1}{|c}{9} & 8 & 7 & 6 & \multicolumn{1}{|c}{5} & 4 & 3 & \multicolumn{1}{|c}{2} & 1 & 0 \\
\hline
0 & 1 & 0 & 0 & 0 & 0 & \multicolumn{1}{|c}{0} & 1 & 1 & 1 & \multicolumn{3}{|c}{Rm} & \multicolumn{3}{|c|}{Rdn} \\
\hline
\end{tabular}


\subsubsubsection{TST (register): Set flags on bitwise AND (p. 466)}

\textbf{Description: }

Effectue un ET logique entre le contenu du registre \texttt{Rn} et le contenu du registre \texttt{Rm}, le résultat n'est pas écrit.\\
Les drapeaux sont mis à jour.\\

\textbf{Assembleur:} T1

\begin{lstlisting}
TST <Rn>,<Rm>
\end{lstlisting}

\textbf{Binaire:}\\

\begin{tabular}{| c c c c c c c c c c c c c c c c |}
\hline
15 & 14 & 13 & 12 & 11 & 10 & \multicolumn{1}{|c}{9} & 8 & 7 & 6 & \multicolumn{1}{|c}{5} & 4 & 3 & \multicolumn{1}{|c}{2} & 1 & 0 \\
\hline
0 & 1 & 0 & 0 & 0 & 0 & \multicolumn{1}{|c}{1} & 0 & 0 & 0 & \multicolumn{3}{|c}{Rm} & \multicolumn{3}{|c|}{Rn} \\
\hline
\end{tabular}



\subsubsubsection{RSB (immediate): Reverse Substract from 0 (p. 372)}

\textbf{Description: }

Soustrait le contenu du registre \texttt{Rn} à l'immédiat 0, écrit le résultat dans le registre \texttt{Rd}.\\
Les drapeaux sont mis à jour.\\

\textbf{Assembleur:} T1

\begin{lstlisting}
RSBS <Rd>,<Rn>,#0
\end{lstlisting}

\textbf{Binaire:}\\

\begin{tabular}{| c c c c c c c c c c c c c c c c |}
\hline
15 & 14 & 13 & 12 & 11 & 10 & \multicolumn{1}{|c}{9} & 8 & 7 & 6 & \multicolumn{1}{|c}{5} & 4 & 3 & \multicolumn{1}{|c}{2} & 1 & 0 \\
\hline
0 & 1 & 0 & 0 & 0 & 0 & \multicolumn{1}{|c}{1} & 0 & 0 & 1 & \multicolumn{3}{|c}{Rn} & \multicolumn{3}{|c|}{Rd} \\
\hline
\end{tabular}



\subsubsubsection{CMP (register): Compare Registers (p. 231)}

\textbf{Description: }

Soustrait le contenu du registre \texttt{Rm} au contenu du registre \texttt{Rn}, le résultat n'est pas écrit.\\
Les drapeaux sont mis à jour.\\

\textbf{Assembleur:} T1

\begin{lstlisting}
CMP <Rn>,<Rm>
\end{lstlisting}

\textbf{Binaire:}\\

\begin{tabular}{| c c c c c c c c c c c c c c c c |}
\hline
15 & 14 & 13 & 12 & 11 & 10 & \multicolumn{1}{|c}{9} & 8 & 7 & 6 & \multicolumn{1}{|c}{5} & 4 & 3 & \multicolumn{1}{|c}{2} & 1 & 0 \\
\hline
0 & 1 & 0 & 0 & 0 & 0 & \multicolumn{1}{|c}{1} & 0 & 1 & 0 & \multicolumn{3}{|c}{Rm} & \multicolumn{3}{|c|}{Rn} \\
\hline
\end{tabular}


\subsubsubsection{CMN (register): Compare Negative (p. 227)}

\textbf{Description: }

Ajoute le contenu du registre \texttt{Rm} au contenu du registre \texttt{Rn}, le résultat n'est pas écrit.\\
Les drapeaux sont mis à jour.\\

\textbf{Assembleur:} T1

\begin{lstlisting}
CMN <Rn>,<Rm>
\end{lstlisting}

\textbf{Binaire:}\\

\begin{tabular}{| c c c c c c c c c c c c c c c c |}
\hline
15 & 14 & 13 & 12 & 11 & 10 & \multicolumn{1}{|c}{9} & 8 & 7 & 6 & \multicolumn{1}{|c}{5} & 4 & 3 & \multicolumn{1}{|c}{2} & 1 & 0 \\
\hline
0 & 1 & 0 & 0 & 0 & 0 & \multicolumn{1}{|c}{1} & 0 & 1 & 1 & \multicolumn{3}{|c}{Rm} & \multicolumn{3}{|c|}{Rn} \\
\hline
\end{tabular}


\subsubsubsection{ORR (register): Logical OR (p. 336)}

\textbf{Description: }

Effectue un OU binaire entre le contenu du registre \texttt{Rdn} et le contenu du registre \texttt{Rm}, écrit le résultat dans le registre \texttt{Rdn}.\\
Les drapeaux sont mis à jour.\\

\textbf{Assembleur:} T1

\begin{lstlisting}
ORRS <Rdn>,<Rm>
\end{lstlisting}

\textbf{Binaire:}\\

\begin{tabular}{| c c c c c c c c c c c c c c c c |}
\hline
15 & 14 & 13 & 12 & 11 & 10 & \multicolumn{1}{|c}{9} & 8 & 7 & 6 & \multicolumn{1}{|c}{5} & 4 & 3 & \multicolumn{1}{|c}{2} & 1 & 0 \\
\hline
0 & 1 & 0 & 0 & 0 & 0 & \multicolumn{1}{|c}{1} & 1 & 0 & 0 & \multicolumn{3}{|c}{Rm} & \multicolumn{3}{|c|}{Rdn} \\
\hline
\end{tabular}


\subsubsubsection{MUL: Multiply Two Registers (p. 324)}

\textbf{Description: }

Multiplie le contenu du registre \texttt{Rn} avec le contenu du registre \texttt{Rdm}, écrit les 32 bits de poids faible du résultat dans le registre \texttt{Rdm}.\\
Les drapeaux sont mis à jour.\\

\textbf{Assembleur:} T1

\begin{lstlisting}
MULS <Rdm>,<Rn>,<Rdm>
\end{lstlisting}

\textbf{Binaire:}\\

\begin{tabular}{| c c c c c c c c c c c c c c c c |}
\hline
15 & 14 & 13 & 12 & 11 & 10 & \multicolumn{1}{|c}{9} & 8 & 7 & 6 & \multicolumn{1}{|c}{5} & 4 & 3 & \multicolumn{1}{|c}{2} & 1 & 0 \\
\hline
0 & 1 & 0 & 0 & 0 & 0 & \multicolumn{1}{|c}{1} & 1 & 0 & 1 & \multicolumn{3}{|c}{Rm} & \multicolumn{3}{|c|}{Rdm} \\
\hline
\end{tabular}


\subsubsubsection{BIC (register): Bit Clear (p. 213)}

\textbf{Description: }

Effectue un ET binaire entre le contenu du registre \texttt{Rdn} et le contenu du registre \texttt{Rm}, écrit le résultat dans le registre \texttt{Rdn}.\\
Les drapeaux sont mis à jour.\\

\textbf{Assembleur:} T1

\begin{lstlisting}
BICS <Rdn>,<Rm>
\end{lstlisting}

\textbf{Binaire:}\\

\begin{tabular}{| c c c c c c c c c c c c c c c c |}
\hline
15 & 14 & 13 & 12 & 11 & 10 & \multicolumn{1}{|c}{9} & 8 & 7 & 6 & \multicolumn{1}{|c}{5} & 4 & 3 & \multicolumn{1}{|c}{2} & 1 & 0 \\
\hline
0 & 1 & 0 & 0 & 0 & 0 & \multicolumn{1}{|c}{1} & 1 & 1 & 0 & \multicolumn{3}{|c}{Rm} & \multicolumn{3}{|c|}{Rdn} \\
\hline
\end{tabular}


\subsubsubsection{MVN (register): Bitwise NOT (p. 328)}

\textbf{Description: }

Effectue un NON binaire sur le contenu du registre \texttt{Rm}, écrit le résultat dans le registre \texttt{Rd}.\\
Les drapeaux sont mis à jour.\\

\textbf{Assembleur:} T1

\begin{lstlisting}
MVNS <Rd>,<Rm>
\end{lstlisting}

\textbf{Binaire:}\\

\begin{tabular}{| c c c c c c c c c c c c c c c c |}
\hline
15 & 14 & 13 & 12 & 11 & 10 & \multicolumn{1}{|c}{9} & 8 & 7 & 6 & \multicolumn{1}{|c}{5} & 4 & 3 & \multicolumn{1}{|c}{2} & 1 & 0 \\
\hline
0 & 1 & 0 & 0 & 0 & 0 & \multicolumn{1}{|c}{1} & 1 & 1 & 1 & \multicolumn{3}{|c}{Rm} & \multicolumn{3}{|c|}{Rd} \\
\hline
\end{tabular}


\subsubsection{Load/Store}

\textbf{Binaire:}\\

\begin{tabular}{| c c c c c c c c c c c c c c c c |}
\hline
15 & 14 & 13 & 12 & \multicolumn{1}{|c}{11} & 10 & 9 & \multicolumn{1}{|c}{8} & 7 & 6 & 5 & 4 & 3 & 2 & 1 & 0 \\
\hline
1 & 0 & 0 & 1 & \multicolumn{3}{|c}{opcode} & \multicolumn{9}{|c|}{} \\
\hline
\end{tabular}

\subsubsubsection{STR (immediate): Store Register (p. 426)}

\textbf{Description: }

Écrit un mot de 32 bits contnue dans le registre \texttt{Rt} à l'adresse mémoire spécifiée.\\
L'adresse mémoire est calculée à partir du contenu du registre \texttt{SP} plus l'immédiat \texttt{imm8}.\\

\textbf{Assembleur:} T2

\begin{lstlisting}
STR <Rt>,[SP,#<imm8>]
\end{lstlisting}

\textbf{Binaire:}\\

\begin{tabular}{| c c c c c c c c c c c c c c c c |}
\hline
15 & 14 & 13 & 12 & \multicolumn{1}{|c}{11} & \multicolumn{1}{|c}{10} & 9 & 8 & \multicolumn{1}{|c}{7} & 6 & 5 & 4 & 3 & 2 & 1 & 0 \\
\hline
1 & 0 & 0 & 1 & \multicolumn{1}{|c}{0} & \multicolumn{3}{|c}{Rt} & \multicolumn{8}{|c|}{imm8} \\
\hline
\end{tabular}


\subsubsubsection{LDR (immediate): Load Register (p. 252)}

\textbf{Description: }

Charge un mot de 32 bits contenu à l'adresse mémoire spécifiée, écrit le résultat dans le registre \texttt{Rt}.\\
L'adresse mémoire est calculée à partir du contenu du registre \texttt{SP} plus l'immédiat \texttt{imm8}.\\

\textbf{Assembleur:} T2

\begin{lstlisting}
LDR <Rt>,[SP{,#<imm8>}]
\end{lstlisting}

\textbf{Binaire:}\\

\begin{tabular}{| c c c c c c c c c c c c c c c c |}
\hline
15 & 14 & 13 & 12 & \multicolumn{1}{|c}{11} & \multicolumn{1}{|c}{10} & 9 & 8 & \multicolumn{1}{|c}{7} & 6 & 5 & 4 & 3 & 2 & 1 & 0 \\
\hline
1 & 0 & 0 & 1 & \multicolumn{1}{|c}{1} & \multicolumn{3}{|c}{Rt} & \multicolumn{8}{|c|}{imm8} \\
\hline
\end{tabular}

\subsubsection{Branch}

\textbf{Binaire:}\\

\begin{tabular}{| c c c c c c c c c c c c c c c c |}
\hline
15 & 14 & 13 & 12 & \multicolumn{1}{|c}{11} & 10 & 9 & 8 & \multicolumn{1}{|c}{7} & 6 & 5 & 4 & 3 & 2 & 1 & 0 \\
\hline
1 & 1 & 0 & 1 & \multicolumn{4}{|c}{opcode} & \multicolumn{8}{|c|}{} \\
\hline
\end{tabular}

\subsubsubsection{B: Conditional Branch (p. 207)}

\textbf{Description: }

Continue l'exécution à partir de l'étiquette \texttt{label} si la condition \texttt{<c>} est vérifiée.\\

\textbf{Assembleur:} T1

\begin{lstlisting}
B<c> <label>
\end{lstlisting}

\textbf{Binaire:}\\

\begin{tabular}{| c c c c c c c c c c c c c c c c |}
\hline
15 & 14 & 13 & 12 & \multicolumn{1}{|c}{11} & 10 & 9 & 8 & \multicolumn{1}{|c}{7} & 6 & 5 & 4 & 3 & 2 & 1 & 0 \\
\hline
1 & 1 & 0 & 1 & \multicolumn{4}{|c}{cond} & \multicolumn{8}{|c|}{imm8} \\
\hline
\end{tabular}

\subsection{Conditions (p. 176)}

\begin{tabular}{| c | c | c | c |}
\hline
\textbf{code} & \textbf{symbole} & \textbf{signification} & \textbf{drapeaux}\\
\hline
0000 & EQ & égalité & Z == 1\\
\hline
0001 & NE & différence & Z == 0\\
\hline
0010 & CS & retenue & C == 1\\
\hline
0011 & CC & pas de retenue & C == 0\\
\hline
0100 & MI & négatif & N == 1\\
\hline
0101 & PL & positif ou nul & N == 0\\
\hline
0110 & VS & dépassement de capacité & V == 1\\
\hline
0111 & VC & pas de dépassement de capacité & V == 0\\
\hline
1000 & HI & supérieur (non signé) & C == 1 et Z == 0\\
\hline
1001 & LS & inférieur ou égal (non signé) & C == 0 et Z == 1\\
\hline
1010 & GE & supérieur ou égal (signé) & N == V\\
\hline
1011 & LT & inférieur (signé) & N != V\\
\hline
1100 & GT & supérieur (signé) & Z == 0 et N == V\\
\hline
1101 & LE & inférieur ou égal (signé) & Z == 1 ou Z != V\\
\hline
1110 & aucun ou AL & toujours vrai & \\
\hline
\end{tabular}


\subsection{Drapeaux (p. 31)}

\begin{itemize}
	\item \texttt{N}: résultat négatif, égal au bit de poids fort du résultat
	\item \texttt{Z}: résultat nul, égal à 1 si le résultat est 0
	\item \texttt{C}: retenue
	\item \texttt{V}: dépassement de capacité
\end{itemize}



\section{Répartition des rôles}
\subsection{Nyan}
\subsubsection{Meow}
\subsubsubsection{Mjau}
MiaouNyanMeowMjau

\section{Décodeur 7 segment}
\subsection{nyan}
\subsubsection{meow}
\subsubsubsection{mjau}
miaounyanmeowmjau

\section{ALU}
\subsection{nyan}
\subsubsection{meow}
\subsubsubsection{mjau}
miaounyanmeowmjau

\section{Banc de registres}
\subsection{nyan}
\subsubsection{meow}
\subsubsubsection{mjau}
miaounyanmeowmjau

\section{Usage de la documentation ARM}
\subsection{nyan}
\subsubsection{meow}
\subsubsubsection{mjau}
miaounyanmeowmjau

\end{document}
