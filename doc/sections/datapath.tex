\documentclass{article}

\usepackage{../parm}
\begin{document}

    \section{Chemin de données}

    \subsection{Description}

    Il s'agit d'interfacer les différents composants CPU, RAM et ROM afin d'obtenir une machine fonctionelle qui puisse exécuter un programme automatiquement.

    Un bouton \texttt{Reset} contrôle l'entrée \texttt{Reset} du processeur afin de remettre à zéro son état et recommencer l'exécution du programme depuis le début.

    Un bouton \texttt{Clock} contrôle l'entrée \texttt{Clk} du processeur et de la RAM afin de contrôler manuellement l'exécution de chacun des instructions.
    Pour une exécution automatique et pour le déploiement sur FPGA, on utilisera le composant \texttt{Horloge} de Logisim.

    \subsection{CPU}

    \subsubsection{Description}

    Le processeur regroupe le contrôleur, le banc de registres et l'ALU.

    À chaque coup d'horloge, l'instruction est décodée par le contrôleur, le calcul est effectué par l'ALU tandis que le banc de registres lui fournit les données et enregistre le résultat.

    \paragraph{}
    Lors d'une instruction \texttt{LDR}, le signal \texttt{Load} du contrôleur est activé, et la donnée en entrée du banc de registres doit être lue à partir de l'entrée \texttt{RAM\_In} plutôt qu'à partir du résultat de l'ALU.

    Lors d'une instruction \texttt{STR}, le signal \texttt{Store} du contrôleur est activé et est passé à la RAM. Les données en sortie \texttt{RAM\_Out} sont fournies par la sortie \texttt{AOut} du banc de registres.

    \paragraph{}
    Lors d'une instruction de la catégorie \textit{Shift, add, sub, mov}, le signal \texttt{DP\_Shift} du contrôleur est activé et l'immédiat \texttt{Imm5} en sortie du contrôleur est utilisé pour l'entrée \texttt{Shift} de l'ALU.

    Lors d'une instruction de la catégorie \textit{Data Processing}, le signal \texttt{DP\_Shift} du contrôleur est désactivé et les 5 bits de poids faible de la sortie \texttt{AOut} sont utilisés pour l'entrée \texttt{Shift} de l'ALU.

    \paragraph{}
    Lors d'une instruction \texttt{MOV} ou \texttt{ADD} et \texttt{SUB} avec immédiat, le signal \texttt{Imm32\_Enable} du contrôleur est activé
    et la donnée à l'entrée \texttt{A} de l'ALU doit être lue à partir de la sortie \texttt{Imm32} du contrôleur plutôt qu'à partir de la sortie \texttt{AOut} du banc de registres.

    \paragraph{}
    Les sorties \texttt{Rm}, \texttt{Rn} et \texttt{Rd} du contrôleur se connectent respectivement aux entrées \texttt{RegA}, \texttt{RegB} et \texttt{RegDest} du banc de registres pour sélectionner les registres correspondant.

    \paragraph{}
    Les sorties \texttt{PC} et \texttt{RAM\_Addr} se connectent respectivement aux sorties \texttt{ROM\_Addr} et \texttt{RAM\_Addr} de ce composant CPU pour sélectionner les adresses de la ROM et de la RAM.

    \subsubsection{Interface}

    \subsubsubsection{Entrées}

    \begin{tabular}{|l|r|l|}
        \hline
        \textbf{Port}   & \textbf{Taille} & \textbf{Description}                   \\
        \hline

        \texttt{RAM\_In} & \texttt{32}     & Données chargées à partir de la RAM    \\
        \hline
        \texttt{ROM\_In} & \texttt{16}     & Instruction chargée à partir de la ROM \\
        \hline
        \texttt{Clk}    & \texttt{1}      & Horloge                                \\
        \hline
        \texttt{Reset}  & \texttt{1}      & Remise à Zero                          \\


        \hline
    \end{tabular}

    \subsubsubsection{Sorties}

    \begin{tabular}{|l|r|l|}
        \hline
        \textbf{Port}        & \textbf{Taille}     & \textbf{Description}                                                                       \\
        \hline

        \hline
        \texttt{ROM\_Addr} & \texttt{8}          & Adresse de la prochaine instruction                                                        \\
        \hline
        \texttt{RAM\_Addr} & \texttt{8}          & Adresse de la donnée à lire/écrire en mémoire                                              \\
        \hline
        \texttt{RAM\_Out}  & \texttt{32}         & Donnée à écrire en mémoire                                                                 \\
        \hline
        \texttt{Store}    & \texttt{1}          & Indique à la RAM de sauvegarder la donnée \texttt{RAM\_Out} à l'adresse \texttt{RAM\_Addr} \\
        \hline
        \texttt{R0-R7}    & \texttt{8\times 32} & Sortie des registres utilisées à des fins de débogage et visualisation du comportement     \\


        \hline
    \end{tabular}

    \subsection{RAM}

    \subsubsection{Description}

    La RAM est la mémoire de données dans laquelle le programme vient stocker le contenu d'un registre ou lire une donnée pour remplir un registre.
    Synchrone, elle ne lit ou n'enregistre les données qu'au coup d'horloge suivant.

    Elle est adressée sur 8 bits et contient des données de 32 bits.

    Le signal \texttt{Load} est définit à 1 de manière à toujours obtenir la données à l'adresse \texttt{Address} sur la sortie \texttt{Data}.

    Lorsque le signal \texttt{Store} est activé, les données présentent sur l'entrée \texttt{Input} sont enregistrées dans la RAM à l'adresse \texttt{Address}.

    \paragraph{Remarque:} le jeu d'instructions ARMv7-M adresse la RAM octet par octet (1 adresse = 8 bits de données) tandis que sous Logisim, le composant RAM est adressé mot par mot (1 adresse = 32 bits de données).
    Cela permet de simplifier la gestion de la RAM puisqu'il est possible de charger des données directement dans les registres 32 bits du CPU.

    \paragraph{Remarque 2:} sous Logisim, le paramètre \texttt{Databus implementation} doit être défini à \texttt{Separate databus for read and write}
    et le paramètre \texttt{Trigger} à \texttt{Flanc Montant} afin de pouvoir déployer le composant sur FPGA.

    \subsubsection{Interface}

    \subsubsubsection{Entrées}

    \begin{tabular}{|l|r|l|}
        \hline
        \textbf{Port}   & \textbf{Taille} & \textbf{Description}                               \\
        \hline

        \texttt{Address} & \texttt{8}      & Adresse à laquelle lire/écrire les données         \\
        \hline
        \texttt{Load}   & \texttt{1}      & Lire les données au prochain coup d'horloge        \\
        \hline
        \texttt{Store}  & \texttt{1}      & Enregistrer les données au prochain coup d'horloge \\
        \hline
        \texttt{Clock}  & \texttt{1}      & Horloge                                            \\
        \hline
        \texttt{Input}  & \texttt{32}     & Données à écrire                                   \\


        \hline
    \end{tabular}

    \subsubsubsection{Sorties}

    \begin{tabular}{|l|r|l|}
        \hline
        \textbf{Port}   & \textbf{Taille} & \textbf{Description} \\
        \hline
        \texttt{Data} & \texttt{32}     & Données lues         \\

        \hline
    \end{tabular}

    \subsection{ROM}

    \subsubsection{Description}

    La ROM est la mémoire d'instruction à partir de laquelle les instructions du programme sont lues.
    Elle est accessible en lecture uniquement et est asynchrone.

    Elle est adressée sur 8 bits et contient des données de 16 bits (largeur d'une instruction Thumb).

    \paragraph{}
    Le programme, assemblé par l'assembleur, devra être chargé dans cette ROM

    \subsubsection{Interface}

    \subsubsubsection{Entrées}

    \begin{tabular}{|l|r|l|}
        \hline
        \textbf{Port}   & \textbf{Taille} & \textbf{Description}               \\
        \hline

        \texttt{Address} & \texttt{8}      & Adresse de l'instruction à charger \\

        \hline
    \end{tabular}

    \subsubsubsection{Sorties}

    \begin{tabular}{|l|r|l|}
        \hline
        \textbf{Port}   & \textbf{Taille} & \textbf{Description} \\
        \hline

        \texttt{Data} & \texttt{16}     & Instruction lue      \\

        \hline
    \end{tabular}


%%\subsection{Entrées/Sorties}

\end{document}