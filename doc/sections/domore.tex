\documentclass{article}

\usepackage{../parm}
\begin{document}

    \section{Pour aller plus loin}

    \subsection{Compilation de code C}
    Avec le jeu d'instructions de ce processeur, il est possible d'exécuter du code C compilé par \texttt{Clang}.
    Il doit cependant rester relativement simple avec une structure bien précise.

    En particulier, on évitera:
    \begin{itemize}
        \item Les appels de fonctions (\texttt{LR}, \texttt{PUSH}, \texttt{POP} non implémentés)
        \item Les variables globales et \texttt{static} (adressage uniquement sur la pile)
    \end{itemize}

    On s'assurera donc:
    \begin{itemize}
        \item D'écrire tout le code dans la fonction \texttt{main()}
        \item De placer toutes les valeurs (y compris celles de comparaison dans les conditions) dans des variables
        \item De déclarer toutes les variables dans le corps de la fonction \texttt{main()}
    \end{itemize}

    La commande à utiliser est la suivante:
    \begin{lstlisting}
clang -S -target arm-none-eabi -mcpu=cortex-m0 -O0 -mfloat-abi=soft main.c
    \end{lstlisting}
    avec \texttt{main.c} le fichier source C.

    Cela crééra un fichier de sortie \texttt{main.s} dont il faudra extraire les instructions assembleur,
    de la directive \texttt{.pad} exclue jusqu'à la ligne \texttt{bx lr} exclue qui correspond au retour de la fonction \texttt{main()}.

    \paragraph{Remarque:} \texttt{bx lr} correspond au retour à la fonction appellante, inexistante ici.
    L'instruction n'est de plus pas implémentée, il suffit donc de supprimer cette ligne.
    Si le programme contient une boucle \texttt{while (1)} principale, la fonction \texttt{main()} ne retourne pas donc l'instruction \texttt{bx lr} sera absente.

    \paragraph{Remarque 2:} depuis la version 9 de \texttt{Clang}, l'addressage mémoire est réalisé différemment en utilisant des instructions \texttt{add <Rd>,SP,\#<imm8>} et \texttt{ldr <Rt>,<Rn>,\#<imm5>} qui ne sont pas prises en charge par l'implémentation.
    Il faut donc s'assurer d'utiliser une version 4 à 8 de \texttt{Clang} (paquet \texttt{clang-8} sous Ubuntu).

    Le code assembleur extrait devra être passé à l'assembleur écrit dans le cadre du projet pour générer le fichier lisible par Logisim et pouvoir l'importer dans la ROM.

    \subsection{Entrées/Sorties}

    Pour pouvoir s'interfacer avec le monde extérieur, un système doit disposer d'entrées/sorties.
    Jusque là, notre processeur ne disposait pas d'entrées, le seul contrôle possible était d'actionner l'horloge et de forcer une remise à zéro.
    Quant aux sorties, elles étaient limitées à la visualisation du contenu des registres.

    Nous souhaitons donc pouvoir dialoguer avec un ou plusieurs périphériques.
    Pour cela, nous allons mettre en place le concept de \textit{Memory-mapped I/O}, qui consiste à réserver une partie de l'espace d'adressage mémoire pour les entrées/sorties.
    Dans notre cas, notre RAM est adressée en 8 bits, on peut donc utiliser un 9\ieme {} bit pour dialoguer avec les entrées/sorties.

    Si l'adresse mémoire est strictement inférieure à 256, les instructions \texttt{LDR} et \texttt{STR} affecteront la RAM.

    Si l'adresse mémoire est supérieure ou égale à 256, les instructions \texttt{LDR} et \texttt{STR} affecteront un banc de registres.
    On ne crééra pas 256 registres, quelques registres sur les premières adresses seront amplement suffisant.

    Nos bus d'entrées/sorties pourront être extraits à partir des registres.
    Par exemple, pour dialoguer sur un bus $I^2C$, on pourra extraire les 2 bits de poids faible du registre 0 pour les connecter aux lignes \texttt{SDA} et \texttt{SDL}.

    Notre processeur ne disposant pas d'interruptions, certains protocoles ne seront pas implémentables.
\end{document}