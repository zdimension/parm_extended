\documentclass{article}

\usepackage{../parm}
\begin{document}

    \section{Pour aller plus loin}

    \subsection{Compilation de code C}
    Avec le jeu d'instructions de ce processeur, il est possible d'exécuter du code C compilé par \texttt{Clang}.
    Il doit cependant rester relativement simple avec une structure bien précise.

    En particulier, on évitera:
    \begin{itemize}
        \item Les appels de fonctions (\texttt{LR}, \texttt{PUSH}, \texttt{POP} non implémentés)
        \item Les variables globales et \texttt{static} (adressage uniquement sur la pile)
        \item Les adressages indirects (donc l'utilisation de tableaux ou de chaînes)
    \end{itemize}

    On s'assurera donc:
    \begin{itemize}
        \item D'écrire tout le code dans la fonction \texttt{void run()}
        \item De déclarer toutes les variables dans le corps de cette dernière
    \end{itemize}
    
    Voici le code de démarrage pour un programme C dans le cadre du projet :
    \lstset{
        basicstyle=\ttfamily,
        keywordstyle=\color[rgb]{0,0,1},
        commentstyle=\color[rgb]{0.133,0.545,0.133},
        stringstyle=\color[rgb]{0.627,0.126,0.941},
        frame=single,
        tabsize=4,
        columns=fixed,
        breaklines=true,
        xleftmargin=0pt,
        postbreak=\mbox{\textcolor{red}{$\hookrightarrow$}\space},
    }
     \begin{lstlisting}[language=C]
#include <parm.h>

void run()
{
	BEGIN();
	// code ici	
	END();
}
    \end{lstlisting}

    La commande à utiliser est la suivante (écrivez la, si vous la copiez depuis le PDF les caractères ne seront pas bons) :
    \begin{lstlisting}
clang -S -target arm-none-eabi -mcpu=cortex-m0 -O0 -mthumb -nostdlib -I./include main.c
    \end{lstlisting}
    avec \texttt{main.c} le fichier source C. Veillez à être dans le dossier code\_c en lançant cette commande.

    Cela crééra un fichier de sortie \texttt{main.s} contenant les instructions assembleur, entourées de directives spéciales qui ne nous intéressent pas ici.

    \paragraph{Remarque 2:} depuis la version 9 de \texttt{Clang}, l'addressage mémoire est réalisé différemment en utilisant des instructions \texttt{add <Rd>,SP,\#<imm8>} et \texttt{ldr <Rt>,<Rn>,\#<imm5>} qui ne sont pas prises en charge par l'implémentation.
    Il faut donc s'assurer d'utiliser une version 4 à 8 de \texttt{Clang} (paquet \texttt{clang-8} sous Ubuntu).

    Le code assembleur devra être passé à l'assembleur écrit dans le cadre du projet pour générer le fichier lisible par Logisim et pouvoir l'importer dans la ROM.

\end{document}