\documentclass{article}

\usepackage{../parm}
\begin{document}

    \section{Assembleur}
    \label{sec:ASM}

    \subsection{Introduction}

    Le code binaire à exécuter est obtenu par l'assemblage d'instructions issues du jeu d'instructions ARMv7 (contre un jeu ARMv6 dans le Cortex-M0 réel).

    Le rôle de l'assembleur est de traduire un programme écrit en langage assembleur dans une représentation que le processeur saura interpréter.

    Ici, le langage assembleur sera l'UAL (\textit{Unified Assembler Language}) de ARM, restreint aux instructions ARMv7 implémentées.
    La représentation des instructions en sortie correspond au codage Thumb des instructions, c'est à dire uniquement sur 16-bits (voir \textit{\ref{sec:ISA}~\nameref{sec:ISA}}).

    Le format de sortie aura la particularité d'être un fichier texte lisible par Logisim.
    Les instructions Thumb devront donc être codées en hexadecimal dans un format décrit ci-après.

    \subsection{Syntaxe}
    Syntaxe UAL:\\
    \texttt{S}: màj des drapeaux\\
    \texttt{<c>}: condition\\
    \texttt{Rm} registre opérande 1\\
    \texttt{Rn}: registre opérande B\\
    \texttt{Rd}: registre destination\\
    \texttt{<immN>}: immédiat sur \texttt{N} bits\\
    \texttt{SP}: registre de pointeur de pile en mémoire\\
    \texttt{opcode}: code de l'instruction, peut occuper jusqu'à la taille indiquée\\
    \texttt{[]}: argument optionnel\\

    Exemple:

    Nous allons partir d'un exemple très simple : 3 variables sont déclarées sur la pile, \textit{a} et \textit{b} ont une valeur qui leur est propre et nous stockons dans \textit{c} le résultat de l'addition \textit{a + b}.
    \newline
    \lstset{
basicstyle=\ttfamily,
keywordstyle=\color[rgb]{0,0,1},
commentstyle=\color[rgb]{0.133,0.545,0.133},
stringstyle=\color[rgb]{0.627,0.126,0.941},
frame=single,
tabsize=4,
columns=fixed,
breaklines=true,
xleftmargin=0pt,
postbreak=\mbox{\textcolor{red}{$\hookrightarrow$}\space},
}
    \begin{minipage}{\linewidth}
        Code C :
\begin{lstlisting}[language=C]
int main() {
	int a, b, c;
	a = 0;
	b = 1;
	c = a + b;
}
\end{lstlisting}
    \end{minipage}
    Code assembleur pour le Cortex-M0 avec un pseudo-code C équivalent, et l'encodage correspondant :\\
\begin{minipage}{\linewidth}
\begin{minipage}{1cm}
            \begin{lstlisting}[xleftmargin=0pt]
b083
2000
9002
2101
9101
9902
9a01
1889
9100
b003
            \end{lstlisting}
  \end{minipage}\quad
  \begin{minipage}{.39\textwidth}
            \begin{lstlisting}[language={[ARM]{Assembler}},xleftmargin=0pt]
sub		sp, #12
movs	r0, #0
str		r0, [sp, #8]
movs	r1, #1
str		r1, [sp, #4]
ldr		r1, [sp, #8]
ldr		r2, [sp, #4]
adds	r1, r1, r2
str		r1, [sp]
add		sp, #12
            \end{lstlisting}
  \end{minipage} \quad
  \begin{minipage}{.49\textwidth}
    \begin{lstlisting}[language=C,xleftmargin=0pt]
sp 		= sp - 3;
r0 		= 0;
sp[2] 	= r0;
r1 		= 1;
sp[1] 	= r1;
r1 		= sp[2];
r2 		= sp[1];
r1 		= r1 + r2;
sp[0] 	= r1;
sp 		= sp + 3;
            \end{lstlisting}
  \end{minipage}
\end{minipage}

    \subsection{Syntaxe Logisim}

    Voici un exemple de fichier lisible par Logisim pour remplir le contenu de la ROM obtenu par assemblage du code assembleur ci-dessus :
    \begin{lstlisting}
v2.0 raw
b083 2000 9002 2101 9101 9902 9a01 1889 9100 b003
    \end{lstlisting}

    On observe dont un entête \texttt{v2.0 raw}, toujours présent sur la première ligne.

    Sur les lignes suivantes, les instructions sont disposées par groupes de 4 caractères hexadecimaux séparés par des espaces, ce qui représente 16 bits.
    On a donc une instruction par groupe.
    Les retours à la ligne sont optionnels.
\end{document}