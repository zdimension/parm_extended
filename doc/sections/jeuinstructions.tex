\documentclass{article}

\usepackage{../parm}
\begin{document}

    \section{Jeu d'instructions (Instruction Set Architecture)}
    \label{sec:ISA}

    Toutes les informations présentes dans cette section proviennent directement du manuel de référence de l'architecture ARMv7-M (\textit{ARM v7-M Architecture Reference Manual}).
    Elles ont été traduites et réorganisées pour en faciliter la lecture.
    En cas de doute, ou pour en savoir plus, les pages du manuel sont indiquées entre parenthèses.

    \subsection{Instructions à implémenter}

    \textbf{Binaire:}

    \begin{tabular}{| c c c c c c c c c c c c c c c c |}
        \hline
        15 & 14 & 13 & 12 & 11 & 10 & \multicolumn{1}{|c}{9} & 8 & 7 & 6 & 5 & 4 & 3 & 2 & 1 & 0 \\
        \hline
        \multicolumn{6}{|c}{opcode} & \multicolumn{10}{|c|}{} \\
        \hline
    \end{tabular}

    \subsubsection{Shift, add, sub, mov}
    \label{subsubsec:ShiftAddSubMov}

    \textbf{Binaire:}

    \begin{tabular}{| c c c c c c c c c c c c c c c c |}
        \hline
        15 & 14 & \multicolumn{1}{|c}{13} & 12 & 11 & 10 & 9 & \multicolumn{1}{|c}{8} & 7 & 6 & 5 & 4 & 3 & 2 & 1 & 0 \\
        \hline
        0 & 0 & \multicolumn{5}{|c}{opcode} & \multicolumn{9}{|c|}{} \\
        \hline
    \end{tabular}


    \subsubsubsection{LSL (immediate): Logical Shift Left (p. 282)}

    \textbf{Description: }

    Décale le contenu du registre \texttt{Rm} vers la gauche d'un nombre de bits donné par l'immédiat \texttt{imm5}, écrit le résultat dans le registre \texttt{Rd}.\\
    Des zéros sont insérés à droite.\\
    Les drapeaux suivants sont mis à jour:\\
    \texttt{N = 1} si \texttt{résultat < 0}, \texttt{N = 0} sinon.\\
    \texttt{Z = 1} si \texttt{résultat = 0}, \texttt{Z = 0} sinon.\\
    \texttt{C = Rm<0 - shift>} avec shift le nombre de décalage.
    Autrement dit, \texttt{C} est égal au dernier bit sortant.

    \textbf{Assembleur:} T1

    \begin{lstlisting}
LSLS <Rd>,<Rm>,#<imm5>
    \end{lstlisting}

    \textbf{Binaire:}

    \begin{tabular}{| c c c c c c c c c c c c c c c c |}
        \hline
        15 & 14 & 13 & \multicolumn{1}{|c}{12} & 11 & \multicolumn{1}{|c}{10} & 9 & 8 & 7 & 6 & \multicolumn{1}{|c}{5} & 4 & 3 & \multicolumn{1}{|c}{2} & 1 & 0 \\
        \hline
        0 & 0 & 0 & \multicolumn{1}{|c}{0} & 0 & \multicolumn{5}{|c|}{imm5} & \multicolumn{3}{|c|}{Rm} & \multicolumn{3}{|c|}{Rd} \\
        \hline
    \end{tabular}

    \subsubsubsection{LSR (immediate): Logical Shift Right (p. 284)}

    \textbf{Description: }

    Décale le contenu du registre \texttt{Rm} vers la droite d'un nombre de bits donné par l'immédiat \texttt{imm5}, écrit le résultat dans le registre \texttt{Rd}.\\
    Des zéros sont insérés à gauche.\\
    Les drapeaux suivants sont mis à jour:\\
    \texttt{N = 1} si \texttt{résultat < 0}, \texttt{N = 0} sinon.\\
    \texttt{Z = 1} si \texttt{résultat = 0}, \texttt{Z = 0} sinon.\\
    \texttt{C = Rm<shift - 1>} avec shift le nombre de décalage.
    Autrement dit, \texttt{C} est égal au dernier bit sortant.

    \textbf{Assembleur:} T1

    \begin{lstlisting}
LSRS <Rd>,<Rm>,#<imm5>
    \end{lstlisting}

    \textbf{Binaire:}

    \begin{tabular}{| c c c c c c c c c c c c c c c c |}
        \hline
        15 & 14 & 13 & \multicolumn{1}{|c}{12} & 11 & \multicolumn{1}{|c}{10} & 9 & 8 & 7 & 6 & \multicolumn{1}{|c}{5} & 4 & 3 & \multicolumn{1}{|c}{2} & 1 & 0 \\
        \hline
        0 & 0 & 0 & \multicolumn{1}{|c}{0} & 1 & \multicolumn{5}{|c|}{imm5} & \multicolumn{3}{|c|}{Rm} & \multicolumn{3}{|c|}{Rd} \\
        \hline
    \end{tabular}


    \subsubsubsection{ASR (immediate): Arithmetic Shift Right (p. 203)}

    \textbf{Description: }

    Décale le contenu du registre \texttt{Rm} vers la droite d'un nombre de bits donné par l'immédiat \texttt{imm5}, écrit le résultat dans le registre \texttt{Rd}.\\
    Le bit de signe de \texttt{Rm} est ré-inséré à gauche.\\
    Les drapeaux suivants sont mis à jour:\\
    \texttt{N = 1} si \texttt{résultat < 0}, \texttt{N = 0} sinon.\\
    \texttt{Z = 1} si \texttt{résultat = 0}, \texttt{Z = 0} sinon.\\
    \texttt{C = Rm<shift - 1>} avec shift le nombre de décalage.
    Autrement dit, \texttt{C} est égal au dernier bit sortant.

    \textbf{Assembleur:} T1

    \begin{lstlisting}
ASRS <Rd>,<Rm>,#<imm5>
    \end{lstlisting}

    \textbf{Binaire:}

    \begin{tabular}{| c c c c c c c c c c c c c c c c |}
        \hline
        15 & 14 & 13 & \multicolumn{1}{|c}{12} & 11 & \multicolumn{1}{|c}{10} & 9 & 8 & 7 & 6 & \multicolumn{1}{|c}{5} & 4 & 3 & \multicolumn{1}{|c}{2} & 1 & 0 \\
        \hline
        0 & 0 & 0 & \multicolumn{1}{|c}{1} & 0 & \multicolumn{5}{|c|}{imm5} & \multicolumn{3}{|c|}{Rm} & \multicolumn{3}{|c|}{Rd} \\
        \hline
    \end{tabular}


    \subsubsubsection{ADD (register): Add register (p. 192)}

    \textbf{Description: }

    Ajoute le contenu du registre \texttt{Rn} au contenu du registre \texttt{Rm}, écrit le résultat dans le registre \texttt{Rd}.\\
    Les drapeaux suivants sont mis à jour:\\
    \texttt{N = 1} si \texttt{résultat < 0}, \texttt{N = 0} sinon.\\
    \texttt{Z = 1} si \texttt{résultat = 0}, \texttt{Z = 0} sinon.\\
    \texttt{C = 1} en cas de dépassement de capacité lors d'une opération non signée.\\
    \texttt{V = 1} en cas de dépassement de capacité lors d'une opération signée.

    \textbf{Assembleur:} T1

    \begin{lstlisting}
ADDS <Rd>,<Rn>,<Rm>
    \end{lstlisting}

    \textbf{Binaire:}

    \begin{tabular}{| c c c c c c c c c c c c c c c c |}
        \hline
        15 & 14 & 13 & \multicolumn{1}{|c}{12} & 11 & \multicolumn{1}{|c}{10} & \multicolumn{1}{|c}{9} & \multicolumn{1}{|c}{8} & 7 & 6 & \multicolumn{1}{|c}{5} & 4 & 3 & \multicolumn{1}{|c}{2} & 1 & 0 \\
        \hline
        0 & 0 & 0 & \multicolumn{1}{|c}{1} & 1 & \multicolumn{1}{|c}{0} & \multicolumn{1}{|c}{0} & \multicolumn{3}{|c|}{Rm} & \multicolumn{3}{|c|}{Rn} & \multicolumn{3}{|c|}{Rd} \\
        \hline
    \end{tabular}


    \subsubsubsection{SUB (register): Substract register (p. 404)}

    \textbf{Description: }
    Soustrait le contenu du registre \texttt{Rm} au contenu du registre \texttt{Rn}, écrit le résultat dans le registre \texttt{Rd}.\\
    Les drapeaux suivants sont mis à jour:\\
    \texttt{N = 1} si \texttt{résultat < 0}, \texttt{N = 0} sinon.\\
    \texttt{Z = 1} si \texttt{résultat = 0}, \texttt{Z = 0} sinon.\\
    \texttt{C = 1} en cas de dépassement de capacité lors d'une opération non signée.\\
    \texttt{V = 1} en cas de dépassement de capacité lors d'une opération signée.

    \textbf{Assembleur:} T1

    \begin{lstlisting}
SUBS <Rd>,<Rn>,<Rm>
    \end{lstlisting}

    \textbf{Binaire:}

    \begin{tabular}{| c c c c c c c c c c c c c c c c |}
        \hline
        15 & 14 & 13 & \multicolumn{1}{|c}{12} & 11 & \multicolumn{1}{|c}{10} & \multicolumn{1}{|c}{9} & \multicolumn{1}{|c}{8} & 7 & 6 & \multicolumn{1}{|c}{5} & 4 & 3 & \multicolumn{1}{|c}{2} & 1 & 0 \\
        \hline
        0 & 0 & 0 & \multicolumn{1}{|c}{1} & 1 & \multicolumn{1}{|c}{0} & \multicolumn{1}{|c}{1} & \multicolumn{3}{|c|}{Rm} & \multicolumn{3}{|c|}{Rn} & \multicolumn{3}{|c|}{Rd} \\
        \hline
    \end{tabular}


    \subsubsubsection{ADD (immediate): Add 3-bit immediate (p. 190)}

    \textbf{Description: }

    Ajoute l'immédiat \texttt{Imm3} au contenu du registre \texttt{Rn}, écrit le résultat dans le registre \texttt{Rd}.\\
    Les drapeaux suivants sont mis à jour:\\
    \texttt{N = 1} si \texttt{résultat < 0}, \texttt{N = 0} sinon.\\
    \texttt{Z = 1} si \texttt{résultat = 0}, \texttt{Z = 0} sinon.\\
    \texttt{C = 1} en cas de dépassement de capacité lors d'une opération non signée.\\
    \texttt{V = 1} en cas de dépassement de capacité lors d'une opération signée.

    \textbf{Assembleur:} T1

    \begin{lstlisting}
ADDS <Rd>,<Rn>,<#imm3>
    \end{lstlisting}

    \textbf{Binaire:}

    \begin{tabular}{| c c c c c c c c c c c c c c c c |}
        \hline
        15 & 14 & 13 & \multicolumn{1}{|c}{12} & 11 & \multicolumn{1}{|c}{10} & \multicolumn{1}{|c}{9} & \multicolumn{1}{|c}{8} & 7 & 6 & \multicolumn{1}{|c}{5} & 4 & 3 & \multicolumn{1}{|c}{2} & 1 & 0 \\
        \hline
        0 & 0 & 0 & \multicolumn{1}{|c}{1} & 1 & \multicolumn{1}{|c}{1} & \multicolumn{1}{|c}{0} & \multicolumn{3}{|c|}{Imm3} & \multicolumn{3}{|c|}{Rn} & \multicolumn{3}{|c|}{Rd} \\
        \hline
    \end{tabular}


    \subsubsubsection{SUB (immediate): Subtract 3-bit immediate (p. 402)}

    \textbf{Description: }

    Soustrait l'immédiat \texttt{Imm3} au contenu du registre \texttt{Rn}, écrit le résultat dans le registre \texttt{Rd}.\\
    Les drapeaux suivants sont mis à jour:\\
    \texttt{N = 1} si \texttt{résultat < 0}, \texttt{N = 0} sinon.\\
    \texttt{Z = 1} si \texttt{résultat = 0}, \texttt{Z = 0} sinon.\\
    \texttt{C = 1} en cas de dépassement de capacité lors d'une opération non signée.\\
    \texttt{V = 1} en cas de dépassement de capacité lors d'une opération signée.

    \textbf{Assembleur:} T1

    \begin{lstlisting}
SUBS <Rd>,<Rn>,#<imm3>
    \end{lstlisting}

    \textbf{Binaire:}

    \begin{tabular}{| c c c c c c c c c c c c c c c c |}
        \hline
        15 & 14 & 13 & \multicolumn{1}{|c}{12} & 11 & \multicolumn{1}{|c}{10} & \multicolumn{1}{|c}{9} & \multicolumn{1}{|c}{8} & 7 & 6 & \multicolumn{1}{|c}{5} & 4 & 3 & \multicolumn{1}{|c}{2} & 1 & 0 \\
        \hline
        0 & 0 & 0 & \multicolumn{1}{|c}{1} & 1 & \multicolumn{1}{|c}{1} & \multicolumn{1}{|c}{1} & \multicolumn{3}{|c|}{Imm3} & \multicolumn{3}{|c|}{Rn} & \multicolumn{3}{|c|}{Rd} \\
        \hline
    \end{tabular}


    \subsubsubsection{MOV (immediate): Move (p. 291)}

    \textbf{Description: }

    Écrit l'immédiat \texttt{imm8} dans le registre \texttt{Rd}.\\
    Les drapeaux suivants sont mis à jour:\\
    \texttt{N = 1} si \texttt{résultat < 0}, \texttt{N = 0} sinon.\\
    \texttt{Z = 1} si \texttt{résultat = 0}, \texttt{Z = 0} sinon.

    \textbf{Assembleur:} T1

    \begin{lstlisting}
MOVS <Rd>,#<imm8>
    \end{lstlisting}

    \textbf{Binaire:}

    \begin{tabular}{| c c c c c c c c c c c c c c c c |}
        \hline
        15 & 14 & 13 & \multicolumn{1}{|c}{12} & 11 & \multicolumn{1}{|c}{10} & 9 & 8 & \multicolumn{1}{|c}{7} & 6 & 5 & 4 & 3 & 2 & 1 & 0 \\
        \hline
        0 & 0 & 1 & \multicolumn{1}{|c}{0} & 0 & \multicolumn{3}{|c|}{Rd} & \multicolumn{8}{|c|}{imm8} \\
        \hline
    \end{tabular}

    \subsubsubsection{CMP (immediate): Compare (p. 223)}

    \textbf{Description: }

    Soustrait l'immédiat \texttt{imm8} au contenu du registre \texttt{Rn}, le résultat n'est pas écrit.\\
    Les drapeaux suivants sont mis à jour:\\
    \texttt{N = 1} si \texttt{résultat < 0}, \texttt{N = 0} sinon.\\
    \texttt{Z = 1} si \texttt{résultat = 0}, \texttt{Z = 0} sinon.\\
    \texttt{C = 1} en cas de dépassement de capacité lors d'une opération non signée.\\
    \texttt{V = 1} en cas de dépassement de capacité lors d'une opération signée.

    \textbf{Assembleur:} T1

    \begin{lstlisting}
CMP <Rd>,#<imm8>
    \end{lstlisting}

    \textbf{Binaire:}

    \begin{tabular}{| c c c c c c c c c c c c c c c c |}
        \hline
        15 & 14 & 13 & \multicolumn{1}{|c}{12} & 11 & \multicolumn{1}{|c}{10} & 9 & 8 & \multicolumn{1}{|c}{7} & 6 & 5 & 4 & 3 & 2 & 1 & 0 \\
        \hline
        0 & 0 & 1 & \multicolumn{1}{|c}{0} & 1 & \multicolumn{3}{|c|}{Rd} & \multicolumn{8}{|c|}{imm8} \\
        \hline
    \end{tabular}

    \subsubsubsection{ADD (immediate): Add 8-bit immediate (p. 190)}

    \textbf{Description: }

    Ajoute l'immédiat \texttt{imm8} au contenu du registre \texttt{Rdn}, écrit le résultat dans celui-ci.\\
    Les drapeaux suivants sont mis à jour:\\
    \texttt{N = 1} si \texttt{résultat < 0}, \texttt{N = 0} sinon.\\
    \texttt{Z = 1} si \texttt{résultat = 0}, \texttt{Z = 0} sinon.\\
    \texttt{C = 1} en cas de dépassement de capacité lors d'une opération non signée.\\
    \texttt{V = 1} en cas de dépassement de capacité lors d'une opération signée.

    \textbf{Assembleur:} T2

    \begin{lstlisting}
ADDS <Rdn>, [<Rdn>,] #<imm8>
    \end{lstlisting}

    \textbf{Binaire:}

    \begin{tabular}{| c c c c c c c c c c c c c c c c |}
        \hline
        15 & 14 & 13 & \multicolumn{1}{|c}{12} & 11 & \multicolumn{1}{|c}{10} & 9 & 8 & \multicolumn{1}{|c}{7} & 6 & 5 & 4 & 3 & 2 & 1 & 0 \\
        \hline
        0 & 0 & 1 & \multicolumn{1}{|c}{1} & 0 & \multicolumn{3}{|c|}{Rd} & \multicolumn{8}{|c|}{imm8} \\
        \hline
    \end{tabular}

    \subsubsubsection{SUB (immediate): Subtract 8-bit immediate (p. 402)}

    \textbf{Description: }

    Soustrait l'immédiat \texttt{imm8} au contenu du registre \texttt{Rdn}, écrit le résultat dans celui-ci.\\
    Les drapeaux suivants sont mis à jour:\\
    \texttt{N = 1} si \texttt{résultat < 0}, \texttt{N = 0} sinon.\\
    \texttt{Z = 1} si \texttt{résultat = 0}, \texttt{Z = 0} sinon.\\
    \texttt{C = 1} en cas de dépassement de capacité lors d'une opération non signée.\\
    \texttt{V = 1} en cas de dépassement de capacité lors d'une opération signée.

    \textbf{Assembleur:} T2

    \begin{lstlisting}
SUBS <Rdn>, [<Rdn>,] #<imm8>
    \end{lstlisting}

    \textbf{Binaire:}

    \begin{tabular}{| c c c c c c c c c c c c c c c c |}
        \hline
        15 & 14 & 13 & \multicolumn{1}{|c}{12} & 11 & \multicolumn{1}{|c}{10} & 9 & 8 & \multicolumn{1}{|c}{7} & 6 & 5 & 4 & 3 & 2 & 1 & 0 \\
        \hline
        0 & 0 & 1 & \multicolumn{1}{|c}{1} & 1 & \multicolumn{3}{|c|}{Rd} & \multicolumn{8}{|c|}{imm8} \\
        \hline
    \end{tabular}

    \subsubsection{Data processing}
    \label{subsubsec:DataProc}

    \textbf{Binaire:}

    \begin{tabular}{| c c c c c c c c c c c c c c c c |}
        \hline
        15 & 14 & 13 & 12 & 11 & 10 & \multicolumn{1}{|c}{9} & 8 & 7 & 6 & \multicolumn{1}{|c}{5} & 4 & 3 & 2 & 1 & 0 \\
        \hline
        0 & 1 & 0 & 0 & 0 & 0 & \multicolumn{4}{|c}{opcode} & \multicolumn{6}{|c|}{} \\
        \hline
    \end{tabular}

    \subsubsubsection{AND (register): Bitwise AND (p. 201)}

    \textbf{Description: }

    Effectue un ET binaire entre le contenu du registre \texttt{Rdn} et le contenu du registre \texttt{Rm}, écrit le résultat dans le registre \texttt{Rdn}.\\
    Les drapeaux suivants sont mis à jour:\\
    \texttt{N = 1} si \texttt{résultat < 0}, \texttt{N = 0} sinon.\\
    \texttt{Z = 1} si \texttt{résultat = 0}, \texttt{Z = 0} sinon.\\
    \texttt{C = 0}

    \textbf{Assembleur:} T1

    \begin{lstlisting}
ANDS <Rdn>,<Rm>
    \end{lstlisting}

    \textbf{Binaire:}

    \begin{tabular}{| c c c c c c c c c c c c c c c c |}
        \hline
        15 & 14 & 13 & 12 & 11 & 10 & \multicolumn{1}{|c}{9} & 8 & 7 & 6 & \multicolumn{1}{|c}{5} & 4 & 3 & \multicolumn{1}{|c}{2} & 1 & 0 \\
        \hline
        0 & 1 & 0 & 0 & 0 & 0 & \multicolumn{1}{|c}{0} & 0 & 0 & 0 & \multicolumn{3}{|c}{Rm} & \multicolumn{3}{|c|}{Rdn} \\
        \hline
    \end{tabular}


    \subsubsubsection{EOR (register): Exclusive OR (p. 233)}

    \textbf{Description: }

    Effectue un OU exclusif binaire entre le contenu du registre \texttt{Rdn} et le contenu du registre \texttt{Rm}, écrit le résultat dans le registre \texttt{Rdn}.\\
    Les drapeaux suivants sont mis à jour:\\
    \texttt{N = 1} si \texttt{résultat < 0}, \texttt{N = 0} sinon.\\
    \texttt{Z = 1} si \texttt{résultat = 0}, \texttt{Z = 0} sinon.\\
    \texttt{C = 0}

    \textbf{Assembleur:} T1

    \begin{lstlisting}
EORS <Rdn>,<Rm>
    \end{lstlisting}

    \textbf{Binaire:}

    \begin{tabular}{| c c c c c c c c c c c c c c c c |}
        \hline
        15 & 14 & 13 & 12 & 11 & 10 & \multicolumn{1}{|c}{9} & 8 & 7 & 6 & \multicolumn{1}{|c}{5} & 4 & 3 & \multicolumn{1}{|c}{2} & 1 & 0 \\
        \hline
        0 & 1 & 0 & 0 & 0 & 0 & \multicolumn{1}{|c}{0} & 0 & 0 & 1 & \multicolumn{3}{|c}{Rm} & \multicolumn{3}{|c|}{Rdn} \\
        \hline
    \end{tabular}


    \subsubsubsection{LSL (register): Logical Shift Left (p. 283)}

    \textbf{Description: }

    Décale le contenu du registre \texttt{Rdn} vers la gauche d'un nombre de bits donné par l'octet inférieur du registre \texttt{Rm}, écrit le résultat dans le registre \texttt{Rdn}.\\
    Des zéros sont insérés à droite.\\
    Les drapeaux suivants sont mis à jour:\\
    \texttt{N = 1} si \texttt{résultat < 0}, \texttt{N = 0} sinon.\\
    \texttt{Z = 1} si \texttt{résultat = 0}, \texttt{Z = 0} sinon.\\
    \texttt{C = Rn<0 - shift>} avec shift le nombre de décalage.
    Autrement dit, \texttt{C} est égal au dernier bit sortant.

    \textbf{Assembleur:} T1

    \begin{lstlisting}
LSLS <Rdn>,<Rm>
    \end{lstlisting}

    \textbf{Binaire:}

    \begin{tabular}{| c c c c c c c c c c c c c c c c |}
        \hline
        15 & 14 & 13 & 12 & 11 & 10 & \multicolumn{1}{|c}{9} & 8 & 7 & 6 & \multicolumn{1}{|c}{5} & 4 & 3 & \multicolumn{1}{|c}{2} & 1 & 0 \\
        \hline
        0 & 1 & 0 & 0 & 0 & 0 & \multicolumn{1}{|c}{0} & 0 & 1 & 0 & \multicolumn{3}{|c}{Rm} & \multicolumn{3}{|c|}{Rdn} \\
        \hline
    \end{tabular}



    \subsubsubsection{LSR (register): Logical Shift Right (p. 285)}

    \textbf{Description: }

    Décale le contenu du registre \texttt{Rdn} vers la droite d'un nombre de bits donné par l'octet inférieur du registre \texttt{Rm}, écrit le résultat dans le registre \texttt{Rdn}.\\
    Des zéros sont insérés à gauche.\\
    Les drapeaux suivants sont mis à jour:\\
    \texttt{N = 1} si \texttt{résultat < 0}, \texttt{N = 0} sinon.\\
    \texttt{Z = 1} si \texttt{résultat = 0}, \texttt{Z = 0} sinon.\\
    \texttt{C = Rn<shift - 1>} avec shift le nombre de décalage.
    Autrement dit, \texttt{C} est égal au dernier bit sortant.

    \textbf{Assembleur:} T1

    \begin{lstlisting}
LSRS <Rdn>,<Rm>
    \end{lstlisting}

    \textbf{Binaire:}

    \begin{tabular}{| c c c c c c c c c c c c c c c c |}
        \hline
        15 & 14 & 13 & 12 & 11 & 10 & \multicolumn{1}{|c}{9} & 8 & 7 & 6 & \multicolumn{1}{|c}{5} & 4 & 3 & \multicolumn{1}{|c}{2} & 1 & 0 \\
        \hline
        0 & 1 & 0 & 0 & 0 & 0 & \multicolumn{1}{|c}{0} & 0 & 1 & 1 & \multicolumn{3}{|c}{Rm} & \multicolumn{3}{|c|}{Rdn} \\
        \hline
    \end{tabular}


    \subsubsubsection{ASR (register): Arithmetic Shift Right (p. 204)}

    \textbf{Description: }

    Décale le contenu du registre \texttt{Rdn} vers la droite d'un nombre de bits donné par l'octet inférieur du registre \texttt{Rm}, écrit le résultat dans le registre \texttt{Rdn}.\\
    Le bit de signe de \texttt{Rdn} est ré-inséré à gauche.\\
    Les drapeaux suivants sont mis à jour:\\
    \texttt{N = 1} si \texttt{résultat < 0}, \texttt{N = 0} sinon.\\
    \texttt{Z = 1} si \texttt{résultat = 0}, \texttt{Z = 0} sinon.\\
    \texttt{C = Rn<shift - 1>} avec shift le nombre de décalage.
    Autrement dit, \texttt{C} est égal au dernier bit sortant.

    \textbf{Assembleur:} T1

    \begin{lstlisting}
ASRS <Rdn>,<Rm>
    \end{lstlisting}

    \textbf{Binaire:}

    \begin{tabular}{| c c c c c c c c c c c c c c c c |}
        \hline
        15 & 14 & 13 & 12 & 11 & 10 & \multicolumn{1}{|c}{9} & 8 & 7 & 6 & \multicolumn{1}{|c}{5} & 4 & 3 & \multicolumn{1}{|c}{2} & 1 & 0 \\
        \hline
        0 & 1 & 0 & 0 & 0 & 0 & \multicolumn{1}{|c}{0} & 1 & 0 & 0 & \multicolumn{3}{|c}{Rm} & \multicolumn{3}{|c|}{Rdn} \\
        \hline
    \end{tabular}


    \subsubsubsection{ADC (register): Add with Carry (p. 188)}

    \textbf{Description: }

    Ajoute le contenu du registre \texttt{Rm} et le drapeau de retenu au contenu du registre \texttt{Rdn}, écrit le résultat dans le registre \texttt{Rdn}.\\
    Les drapeaux suivants sont mis à jour:\\
    \texttt{N = 1} si \texttt{résultat < 0}, \texttt{N = 0} sinon.\\
    \texttt{Z = 1} si \texttt{résultat = 0}, \texttt{Z = 0} sinon.\\
    \texttt{C = 1} en cas de dépassement de capacité lors d'une opération non signée.\\
    \texttt{V = 1} en cas de dépassement de capacité lors d'une opération signée.

    \textbf{Assembleur:} T1

    \begin{lstlisting}
ADCS <Rdn>,<Rm>
    \end{lstlisting}

    \textbf{Binaire:}

    \begin{tabular}{| c c c c c c c c c c c c c c c c |}
        \hline
        15 & 14 & 13 & 12 & 11 & 10 & \multicolumn{1}{|c}{9} & 8 & 7 & 6 & \multicolumn{1}{|c}{5} & 4 & 3 & \multicolumn{1}{|c}{2} & 1 & 0 \\
        \hline
        0 & 1 & 0 & 0 & 0 & 0 & \multicolumn{1}{|c}{0} & 1 & 0 & 1 & \multicolumn{3}{|c}{Rm} & \multicolumn{3}{|c|}{Rdn} \\
        \hline
    \end{tabular}

    \subsubsubsection{SBC (register): Substract with Carry (p. 347)}
    \label{subsubsubsec:SBC}

    \textbf{Description: }

    Soustrait le contenu du registre \texttt{Rm} et le complément du drapeau de retenu au contenu du registre \texttt{Rdn}, écrit le résultat dans le registre \texttt{Rdn}.\\
    Les drapeaux suivants sont mis à jour:\\
    \texttt{N = 1} si \texttt{résultat < 0}, \texttt{N = 0} sinon.\\
    \texttt{Z = 1} si \texttt{résultat = 0}, \texttt{Z = 0} sinon.\\
    \texttt{C = 1} en cas de dépassement de capacité lors d'une opération non signée.\\
    \texttt{V = 1} en cas de dépassement de capacité lors d'une opération signée.

    \textbf{Assembleur:} T1

    \begin{lstlisting}
SBCS <Rdn>,<Rm>
    \end{lstlisting}

    \textbf{Binaire:}

    \begin{tabular}{| c c c c c c c c c c c c c c c c |}
        \hline
        15 & 14 & 13 & 12 & 11 & 10 & \multicolumn{1}{|c}{9} & 8 & 7 & 6 & \multicolumn{1}{|c}{5} & 4 & 3 & \multicolumn{1}{|c}{2} & 1 & 0 \\
        \hline
        0 & 1 & 0 & 0 & 0 & 0 & \multicolumn{1}{|c}{0} & 1 & 1 & 0 & \multicolumn{3}{|c}{Rm} & \multicolumn{3}{|c|}{Rdn} \\
        \hline
    \end{tabular}



    \subsubsubsection{ROR (register): Rotate Right (p. 339)}

    \textbf{Description: }

    Pivote le contenu du registre \texttt{Rdn} vers la droite d'un nombre de bits donné par l'octet inférieur du registre \texttt{Rm}, écrit le résultat dans le registre \texttt{Rdn}.\\
    Les bits de \texttt{Rdn} sortant à droite sont ré-insérés à gauche.\\
    Les drapeaux suivants sont mis à jour:\\
    \texttt{N = 1} si \texttt{résultat < 0}, \texttt{N = 0} sinon.\\
    \texttt{Z = 1} si \texttt{résultat = 0}, \texttt{Z = 0} sinon.\\
    \texttt{C = résultat<31>}.
    Autrement dit, \texttt{C} est égal au dernier bit sortant.

    \textbf{Assembleur:} T1

    \begin{lstlisting}
RORS <Rdn>,<Rm>
    \end{lstlisting}

    \textbf{Binaire:}

    \begin{tabular}{| c c c c c c c c c c c c c c c c |}
        \hline
        15 & 14 & 13 & 12 & 11 & 10 & \multicolumn{1}{|c}{9} & 8 & 7 & 6 & \multicolumn{1}{|c}{5} & 4 & 3 & \multicolumn{1}{|c}{2} & 1 & 0 \\
        \hline
        0 & 1 & 0 & 0 & 0 & 0 & \multicolumn{1}{|c}{0} & 1 & 1 & 1 & \multicolumn{3}{|c}{Rm} & \multicolumn{3}{|c|}{Rdn} \\
        \hline
    \end{tabular}


    \subsubsubsection{TST (register): Set flags on bitwise AND (p. 420)}

    \textbf{Description: }

    Effectue un ET logique entre le contenu du registre \texttt{Rn} et le contenu du registre \texttt{Rm}, le résultat n'est pas écrit.\\
    Les drapeaux suivants sont mis à jour:\\
    \texttt{N = 1} si \texttt{résultat < 0}, \texttt{N = 0} sinon.\\
    \texttt{Z = 1} si \texttt{résultat = 0}, \texttt{Z = 0} sinon.\\
    \texttt{C = 0}

    \textbf{Assembleur:} T1

    \begin{lstlisting}
TST <Rn>,<Rm>
    \end{lstlisting}

    \textbf{Binaire:}

    \begin{tabular}{| c c c c c c c c c c c c c c c c |}
        \hline
        15 & 14 & 13 & 12 & 11 & 10 & \multicolumn{1}{|c}{9} & 8 & 7 & 6 & \multicolumn{1}{|c}{5} & 4 & 3 & \multicolumn{1}{|c}{2} & 1 & 0 \\
        \hline
        0 & 1 & 0 & 0 & 0 & 0 & \multicolumn{1}{|c}{1} & 0 & 0 & 0 & \multicolumn{3}{|c}{Rm} & \multicolumn{3}{|c|}{Rn} \\
        \hline
    \end{tabular}



    \subsubsubsection{RSB (immediate): Reverse Substract from 0 (p. 341)}

    \textbf{Description: }

    Soustrait le contenu du registre \texttt{Rn} à l'immédiat 0, écrit le résultat dans le registre \texttt{Rd}.\\
    Les drapeaux suivants sont mis à jour:\\
    \texttt{N = 1} si \texttt{résultat < 0}, \texttt{N = 0} sinon.\\
    \texttt{Z = 1} si \texttt{résultat = 0}, \texttt{Z = 0} sinon.\\
    \texttt{C = 0}.\\
    \texttt{V = 0}.

    \textbf{Assembleur:} T1

    \begin{lstlisting}
RSBS <Rd>,<Rn>,#0
    \end{lstlisting}

    \textbf{Binaire:}

    \begin{tabular}{| c c c c c c c c c c c c c c c c |}
        \hline
        15 & 14 & 13 & 12 & 11 & 10 & \multicolumn{1}{|c}{9} & 8 & 7 & 6 & \multicolumn{1}{|c}{5} & 4 & 3 & \multicolumn{1}{|c}{2} & 1 & 0 \\
        \hline
        0 & 1 & 0 & 0 & 0 & 0 & \multicolumn{1}{|c}{1} & 0 & 0 & 1 & \multicolumn{3}{|c}{Rn} & \multicolumn{3}{|c|}{Rd} \\
        \hline
    \end{tabular}



    \subsubsubsection{CMP (register): Compare Registers (p. 224)}

    \textbf{Description: }

    Soustrait le contenu du registre \texttt{Rm} au contenu du registre \texttt{Rn}, le résultat n'est pas écrit.\\
    Les drapeaux suivants sont mis à jour:\\
    \texttt{N = 1} si \texttt{résultat < 0}, \texttt{N = 0} sinon.\\
    \texttt{Z = 1} si \texttt{résultat = 0}, \texttt{Z = 0} sinon.\\
    \texttt{C = 1} en cas de dépassement de capacité lors d'une opération non signée.\\
    \texttt{V = 1} en cas de dépassement de capacité lors d'une opération signée.

    \textbf{Assembleur:} T1

    \begin{lstlisting}
CMP <Rn>,<Rm>
    \end{lstlisting}

    \textbf{Binaire:}

    \begin{tabular}{| c c c c c c c c c c c c c c c c |}
        \hline
        15 & 14 & 13 & 12 & 11 & 10 & \multicolumn{1}{|c}{9} & 8 & 7 & 6 & \multicolumn{1}{|c}{5} & 4 & 3 & \multicolumn{1}{|c}{2} & 1 & 0 \\
        \hline
        0 & 1 & 0 & 0 & 0 & 0 & \multicolumn{1}{|c}{1} & 0 & 1 & 0 & \multicolumn{3}{|c}{Rm} & \multicolumn{3}{|c|}{Rn} \\
        \hline
    \end{tabular}


    \subsubsubsection{CMN (register): Compare Negative (p. 222)}

    \textbf{Description: }

    Ajoute le contenu du registre \texttt{Rm} au contenu du registre \texttt{Rn}, le résultat n'est pas écrit.\\
    Les drapeaux suivants sont mis à jour:\\
    \texttt{N = 1} si \texttt{résultat < 0}, \texttt{N = 0} sinon.\\
    \texttt{Z = 1} si \texttt{résultat = 0}, \texttt{Z = 0} sinon.\\
    \texttt{C = 1} en cas de dépassement de capacité lors d'une opération non signée.\\
    \texttt{V = 1} en cas de dépassement de capacité lors d'une opération signée.

    \textbf{Assembleur:} T1

    \begin{lstlisting}
CMN <Rn>,<Rm>
    \end{lstlisting}

    \textbf{Binaire:}

    \begin{tabular}{| c c c c c c c c c c c c c c c c |}
        \hline
        15 & 14 & 13 & 12 & 11 & 10 & \multicolumn{1}{|c}{9} & 8 & 7 & 6 & \multicolumn{1}{|c}{5} & 4 & 3 & \multicolumn{1}{|c}{2} & 1 & 0 \\
        \hline
        0 & 1 & 0 & 0 & 0 & 0 & \multicolumn{1}{|c}{1} & 0 & 1 & 1 & \multicolumn{3}{|c}{Rm} & \multicolumn{3}{|c|}{Rn} \\
        \hline
    \end{tabular}


    \subsubsubsection{ORR (register): Logical OR (p. 310)}

    \textbf{Description: }

    Effectue un OU binaire entre le contenu du registre \texttt{Rdn} et le contenu du registre \texttt{Rm}, écrit le résultat dans le registre \texttt{Rdn}.\\
    Les drapeaux suivants sont mis à jour:\\
    \texttt{N = 1} si \texttt{résultat < 0}, \texttt{N = 0} sinon.\\
    \texttt{Z = 1} si \texttt{résultat = 0}, \texttt{Z = 0} sinon.\\
    \texttt{C = 0}.

    \textbf{Assembleur:} T1

    \begin{lstlisting}
ORRS <Rdn>,<Rm>
    \end{lstlisting}

    \textbf{Binaire:}

    \begin{tabular}{| c c c c c c c c c c c c c c c c |}
        \hline
        15 & 14 & 13 & 12 & 11 & 10 & \multicolumn{1}{|c}{9} & 8 & 7 & 6 & \multicolumn{1}{|c}{5} & 4 & 3 & \multicolumn{1}{|c}{2} & 1 & 0 \\
        \hline
        0 & 1 & 0 & 0 & 0 & 0 & \multicolumn{1}{|c}{1} & 1 & 0 & 0 & \multicolumn{3}{|c}{Rm} & \multicolumn{3}{|c|}{Rdn} \\
        \hline
    \end{tabular}


    \subsubsubsection{MUL: Multiply Two Registers (p. 302)}

    \textbf{Description: }

    Multiplie le contenu du registre \texttt{Rn} avec le contenu du registre \texttt{Rdm}, écrit les 32 bits de poids faible du résultat dans le registre \texttt{Rdm}.\\
    Les drapeaux suivants sont mis à jour:\\
    \texttt{N = 1} si \texttt{résultat < 0}, \texttt{N = 0} sinon.\\
    \texttt{Z = 1} si \texttt{résultat = 0}, \texttt{Z = 0} sinon.

    \textbf{Assembleur:} T1

    \begin{lstlisting}
MULS <Rdm>,<Rn>,<Rdm>
    \end{lstlisting}

    \textbf{Binaire:}

    \begin{tabular}{| c c c c c c c c c c c c c c c c |}
        \hline
        15 & 14 & 13 & 12 & 11 & 10 & \multicolumn{1}{|c}{9} & 8 & 7 & 6 & \multicolumn{1}{|c}{5} & 4 & 3 & \multicolumn{1}{|c}{2} & 1 & 0 \\
        \hline
        0 & 1 & 0 & 0 & 0 & 0 & \multicolumn{1}{|c}{1} & 1 & 0 & 1 & \multicolumn{3}{|c}{Rn} & \multicolumn{3}{|c|}{Rdm} \\
        \hline
    \end{tabular}


    \subsubsubsection{BIC (register): Bit Clear (p. 210)}

    \textbf{Description: }

    Effectue un ET binaire entre le contenu du registre \texttt{Rdn} et le complément du contenu du registre \texttt{Rm}, écrit le résultat dans le registre \texttt{Rdn}.\\
    Les drapeaux suivants sont mis à jour:\\
    \texttt{N = 1} si \texttt{résultat < 0}, \texttt{N = 0} sinon.\\
    \texttt{Z = 1} si \texttt{résultat = 0}, \texttt{Z = 0} sinon.\\
    \texttt{C = 0}.

    \textbf{Assembleur:} T1

    \begin{lstlisting}
BICS <Rdn>,<Rm>
    \end{lstlisting}

    \textbf{Binaire:}

    \begin{tabular}{| c c c c c c c c c c c c c c c c |}
        \hline
        15 & 14 & 13 & 12 & 11 & 10 & \multicolumn{1}{|c}{9} & 8 & 7 & 6 & \multicolumn{1}{|c}{5} & 4 & 3 & \multicolumn{1}{|c}{2} & 1 & 0 \\
        \hline
        0 & 1 & 0 & 0 & 0 & 0 & \multicolumn{1}{|c}{1} & 1 & 1 & 0 & \multicolumn{3}{|c}{Rm} & \multicolumn{3}{|c|}{Rdn} \\
        \hline
    \end{tabular}


    \subsubsubsection{MVN (register): Bitwise NOT (p. 304)}

    \textbf{Description: }

    Effectue un NON binaire sur le contenu du registre \texttt{Rm}, écrit le résultat dans le registre \texttt{Rd}.\\
    Les drapeaux suivants sont mis à jour:\\
    \texttt{N = 1} si \texttt{résultat < 0}, \texttt{N = 0} sinon.\\
    \texttt{Z = 1} si \texttt{résultat = 0}, \texttt{Z = 0} sinon.\\
    \texttt{C = 0}.

    \textbf{Assembleur:} T1

    \begin{lstlisting}
MVNS <Rd>,<Rm>
    \end{lstlisting}

    \textbf{Binaire:}

    \begin{tabular}{| c c c c c c c c c c c c c c c c |}
        \hline
        15 & 14 & 13 & 12 & 11 & 10 & \multicolumn{1}{|c}{9} & 8 & 7 & 6 & \multicolumn{1}{|c}{5} & 4 & 3 & \multicolumn{1}{|c}{2} & 1 & 0 \\
        \hline
        0 & 1 & 0 & 0 & 0 & 0 & \multicolumn{1}{|c}{1} & 1 & 1 & 1 & \multicolumn{3}{|c}{Rm} & \multicolumn{3}{|c|}{Rd} \\
        \hline
    \end{tabular}

    \subsubsection{Load/Store}
    \label{subsubsec:LoadStore}

    \textbf{Binaire:}

    \begin{tabular}{| c c c c c c c c c c c c c c c c |}
        \hline
        15 & 14 & 13 & 12 & \multicolumn{1}{|c}{11} & 10 & 9 & \multicolumn{1}{|c}{8} & 7 & 6 & 5 & 4 & 3 & 2 & 1 & 0 \\
        \hline
        1 & 0 & 0 & 1 & \multicolumn{3}{|c}{opcode} & \multicolumn{9}{|c|}{} \\
        \hline
    \end{tabular}

    \subsubsubsection{STR (immediate): Store Register (p. 386)}

    \textbf{Description: }

    Écrit un mot de 32 bits contenu dans le registre \texttt{Rt} à l'adresse mémoire spécifiée.\\
    L'adresse mémoire est calculée à partir du contenu du registre \texttt{SP} plus l'immédiat \texttt{imm8} divisé par 4.

    \textbf{Assembleur:} T2

    \begin{lstlisting}
STR <Rt>,[SP,#<imm8>]
    \end{lstlisting}

    \textbf{Binaire:}

    \begin{tabular}{| c c c c c c c c c c c c c c c c |}
        \hline
        15 & 14 & 13 & 12 & \multicolumn{1}{|c}{11} & \multicolumn{1}{|c}{10} & 9 & 8 & \multicolumn{1}{|c}{7} & 6 & 5 & 4 & 3 & 2 & 1 & 0 \\
        \hline
        1 & 0 & 0 & 1 & \multicolumn{1}{|c}{0} & \multicolumn{3}{|c}{Rt} & \multicolumn{8}{|c|}{imm8 / 4} \\
        \hline
    \end{tabular}


    \subsubsubsection{LDR (immediate): Load Register (p. 246)}

    \textbf{Description: }

    Charge un mot de 32 bits contenu à l'adresse mémoire spécifiée, écrit le résultat dans le registre \texttt{Rt}.\\
    L'adresse mémoire est calculée à partir du contenu du registre \texttt{SP} plus l'immédiat \texttt{imm8} divisé par 4.

    \textbf{Assembleur:} T2

    \begin{lstlisting}
LDR <Rt>,[SP{,#<imm8>}]
    \end{lstlisting}

    \textbf{Binaire:}

    \begin{tabular}{| c c c c c c c c c c c c c c c c |}
        \hline
        15 & 14 & 13 & 12 & \multicolumn{1}{|c}{11} & \multicolumn{1}{|c}{10} & 9 & 8 & \multicolumn{1}{|c}{7} & 6 & 5 & 4 & 3 & 2 & 1 & 0 \\
        \hline
        1 & 0 & 0 & 1 & \multicolumn{1}{|c}{1} & \multicolumn{3}{|c}{Rt} & \multicolumn{8}{|c|}{imm8 / 4} \\
        \hline
    \end{tabular}

    \subsubsection{Miscellaneous 16-bit instructions}
    \label{subsubsec:MiscInstr}

    \textbf{Binaire:}

    \begin{tabular}{| c c c c c c c c c c c c c c c c |}
        \hline
        15 & 14 & 13 & 12 & \multicolumn{1}{|c}{11} & 10 & 9 & 8 & 7 & 6 & 5 & \multicolumn{1}{|c}{4} & 3 & 2 & 1 & 0 \\
        \hline
        1 & 0 & 1 & 1 & \multicolumn{7}{|c}{opcode} & \multicolumn{5}{|c|}{} \\
        \hline
    \end{tabular}

    \subsubsubsection{ADD (SP plus immediate): Add Immediate to SP (p. 194)}

    \textbf{Description: }

    Ajoute l'immédiat \texttt{imm7} divisé par 4 à la valeur du registre \texttt{SP}, écrit le résultat dans le registre \texttt{SP}.
    Les drapeaux ne sont pas mis à jour.

    \textbf{Assembleur:} T2

    \begin{lstlisting}
ADD [SP,]SP,#<imm7>
    \end{lstlisting}

    \textbf{Binaire:}

    \begin{tabular}{| c c c c c c c c c c c c c c c c |}
        \hline
        15 & 14 & 13 & 12 & \multicolumn{1}{|c}{11} & 10 & 9 & 8 & \multicolumn{1}{|c}{7} & \multicolumn{1}{|c}{6} & 5 & 4 & 3 & 2 & 1 & 0 \\
        \hline
        1 & 0 & 1 & 1 & \multicolumn{1}{|c}{0} & 0 & 0 & 0 & \multicolumn{1}{|c}{0} & \multicolumn{7}{|c|}{imm7 / 4} \\
        \hline
    \end{tabular}

    \subsubsubsection{SUB (SP minus immediate): Subtract Immediate from SP (p. 402)}

    \textbf{Description: }

    Soustrait l'immédiat \texttt{imm7} divisé par 4 à la valeur du registre \texttt{SP}, écrit le résultat dans le registre \texttt{SP}.
    Les drapeaux ne sont pas mis à jour.

    \textbf{Assembleur:} T1

    \begin{lstlisting}
SUB [SP,]SP,#<imm7>
    \end{lstlisting}

    \textbf{Binaire:}

    \begin{tabular}{| c c c c c c c c c c c c c c c c |}
        \hline
        15 & 14 & 13 & 12 & \multicolumn{1}{|c}{11} & 10 & 9 & 8 & \multicolumn{1}{|c}{7} & \multicolumn{1}{|c}{6} & 5 & 4 & 3 & 2 & 1 & 0 \\
        \hline
        1 & 0 & 1 & 1 & \multicolumn{1}{|c}{0} & 0 & 0 & 0 & \multicolumn{1}{|c}{1} & \multicolumn{7}{|c|}{imm7 / 4} \\
        \hline
    \end{tabular}

    \subsubsection{Branch}
    \label{subsubsec:Branching}

    Dans le code assembleur, les instructions de branchement ont pour opérande des étiquettes, pointant vers d'autres endroits dans le code. Lors de l'encodage, la valeur de l'immédiat (imm8 ou imm11) doit être égale à $N_{cible} - N_{source} - 3$, où N représente le numéro de l'instruction dans le programme (indexée à 0). Cette valeur peut être négative (si on veut sauter à une instruction présente plus tôt dans le programme) et sera donc encodée comme valeur signée en complément à 2.

    \subsubsubsection{B: Conditional Branch (p. 205)}

    \textbf{Description: }

    Continue l'exécution à partir de l'étiquette \texttt{label} si la condition \texttt{<c>} est vérifiée.

    \textbf{Assembleur:} T1

    \begin{lstlisting}
B<c> <label>
    \end{lstlisting}

    \textbf{Binaire:}

    \begin{tabular}{| c c c c c c c c c c c c c c c c |}
        \hline
        15 & 14 & 13 & 12 & \multicolumn{1}{|c}{11} & 10 & 9 & 8 & \multicolumn{1}{|c}{7} & 6 & 5 & 4 & 3 & 2 & 1 & 0 \\
        \hline
        1 & 1 & 0 & 1 & \multicolumn{4}{|c}{cond} & \multicolumn{8}{|c|}{imm8} \\
        \hline
    \end{tabular}

    \subsubsubsection{B: Unconditional Branch (p. 205)}

    %\textbf{Remarque : } Cette instruction est \textbf{optionnelle} dans le cadre du projet, en effet, un branchement inconditionnel peut être effectué via l'instruction \texttt{B<c>} et la condition \texttt{1110}, mais dans ce cas il ne sera pas possible d'effectuer des sauts de plus de 128 instructions. Ainsi, si vous ne supportez pas l'instruction \texttt{B}, certains programmes d'exemple ne fonctionneront pas. Si vous décidez de ne pas implémenter cette instruction, faites en sorte que votre assembleur lise \texttt{B label} comme \texttt{BAL label}.

    \textbf{Description: }

    Continue l'exécution à partir de l'étiquette \texttt{label}.

    \textbf{Assembleur:} T2

    \begin{lstlisting}
B <label>
    \end{lstlisting}

    \textbf{Binaire:}

    \begin{tabular}{| c c c c c c c c c c c c c c c c |}
        \hline
        15 & 14 & 13 & 12 & 11 & \multicolumn{1}{|c}{10} & 9 & 8 & 7 & 6 & 5 & 4 & 3 & 2 & 1 & 0 \\
        \hline
        1 & 1 & 1 & 0 & 0 & \multicolumn{11}{|c|}{imm11} \\
        \hline
    \end{tabular}

    \subsection{Conditions (p. 178)}
    \label{subsec:CondFlags}

    \begin{tabular}{| c | c | c | c |}
        \hline
        \textbf{code} & \textbf{symbole} & \textbf{signification}         & \textbf{drapeaux} \\
        \hline
        0000          & EQ               & égalité                        & Z == 1            \\
        \hline
        0001          & NE               & différence                     & Z == 0            \\
        \hline
        0010          & CS ou HS         & retenue                        & C == 1            \\
        \hline
        0011          & CC ou LO         & pas de retenue                 & C == 0            \\
        \hline
        0100          & MI               & négatif                        & N == 1            \\
        \hline
        0101          & PL               & positif ou nul                 & N == 0            \\
        \hline
        0110          & VS               & dépassement de capacité        & V == 1            \\
        \hline
        0111          & VC               & pas de dépassement de capacité & V == 0            \\
        \hline
        1000          & HI               & supérieur (non signé)          & C == 1 et Z == 0  \\
        \hline
        1001          & LS               & inférieur ou égal (non signé)  & C == 0 ou Z == 1  \\
        \hline
        1010          & GE               & supérieur ou égal (signé)      & N == V            \\
        \hline
        1011          & LT               & inférieur (signé)              & N != V            \\
        \hline
        1100          & GT               & supérieur (signé)              & Z == 0 et N == V  \\
        \hline
        1101          & LE               & inférieur ou égal (signé)      & Z == 1 ou N != V  \\
        \hline
        1110          & AL             & toujours vrai                  &                   \\
        \hline
    \end{tabular}

    \textbf{Remarque :} Par complétude, vous pouvez considérer que la condition 1111 correspond à "toujours faux". En réalité, cette condition n'est jamais utilisée, mais lors de la réalisation de circuits il est déconseillé de laisser des entrées non reliées.

    \subsection{Drapeaux (p. 31)}
    \label{subsec:Flags}

    \begin{itemize}
        \item \texttt{N}: résultat négatif, égal au bit de poids fort du résultat
        \item \texttt{Z}: résultat nul, égal à 1 si le résultat est 0
        \item \texttt{C}: retenue
        \item \texttt{V}: dépassement de capacité
    \end{itemize}


\end{document}