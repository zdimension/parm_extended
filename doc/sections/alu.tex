\section{ALU}

\subsection{Description}

	L'unité arithmétique et logique est l'élèment qui se charge des calculs au sein du processeur. Les ALU les plus basiques ne font que des opérations sur des entiers cependant on trouve des ALU spécialisées. Les calculs sur ces dernières vont des opérations à virgule flottante jusqu'à des calculs plus complexes tels que des racines carrées, des logarithmes, des sinus ou cosinus...

	Une ALU comporte deux entrées amenant les données à traiter. Une troisième entrée permet de désigner le calcul à effectuer. En sortie on retrouvera le résultat de l'opération ainsi que des drapeaux. Ces drapeaux représentent une série d'état à la suite d'un calcul : un résultat négatif, un résultat nul, un débordement ou encore une retenue. 

\subsection{Interface}

\subsubsection{Entrées}

\begin{tabular}{|l|r|l|}
\hline
\textbf{Port}		& \textbf{Taille} & \textbf{Description}\\
\hline

\texttt{A}		& \texttt{32} & Registre première opérande\\
\hline
\texttt{B}		& \texttt{32} & Registre seconde opérande\\
\hline
\texttt{Shift}		&  \texttt{5} & Immédiat nombre de décalage\\
\hline
\texttt{CarryIn}	&  \texttt{1} & Signal retenu entrente\\
\hline
\texttt{Codop}		&  \texttt{4} & Code opération ALU\\

\hline
\end{tabular}


\subsubsection{Sorties}

\begin{tabular}{|l|r|l|}
\hline 
\textbf{Port} & \textbf{Taille} & \textbf{Description}\\
\hline

\texttt{S}	& \texttt{32} & Registre résultat\\
\hline
\texttt{Flags}	&  \texttt{4} & Registre drapeaux, ordre: \texttt{NZCV}\\

\hline
\end{tabular}


\subsection{Opérations}
\label{subsec:Opcodes}
Ces opérations de l'ALU correspondent exactement aux instructions de la catégorie \textit{Data Processing}. En cas de doute, se référer à \textit{\ref{sec:ISA}~\nameref{sec:ISA}}.

\begin{tabular}{|r|c|l|l|}
\hline
\textbf{Codop}  & \textbf{Opération}	& \textbf{Instructions} & \textbf{Description}\\
\hline

$0000$ & \texttt{A and B}			& AND			&\\
\hline
$0001$ & \texttt{A xor B}			& EOR			&\\
\hline
$0010$ & \texttt{B << Shift}			& LSL			&\\
\hline
$0011$ & \texttt{B >> Shift}			& LSR			&\\
\hline
$0100$ & \texttt{B >> Shift (arith)}		& ASR			&\\
\hline
$0101$ & \texttt{A + B + CarryIn}		& ADC			&\\
\hline
$0110$ & \texttt{A – B + CarryIn – 1}		& SBC			&\\
\hline
$0111$ & \texttt{B >> Shift (rot)}		& ROR			&\\
\hline
$1000$ & \texttt{A and B}			& TST			& Résultat perdu, seuls les drapeaux sont mis à jour\\
\hline
$1001$ & \texttt{0 – B}				& RSB			&\\
\hline
$1010$ & \texttt{A – B}				& CMP			& Résultat perdu, seuls les drapeaux sont mis à jour\\
\hline
$1011$ & \texttt{A + B}				& CMN			& Résultat perdu, seuls les drapeaux sont mis à jour\\
\hline
$1100$ & \texttt{A or B}			& ORR			&\\
\hline
$1101$ & \texttt{A * B}				& MUL			&\\
\hline
$1110$ & \texttt{A and not B}			& BIC			&\\
\hline
$1111$ & \texttt{Not B}				& MVN			&\\
\hline
\end{tabular}

\subsection{Drapeaux}

Ces drapeaux de l'ALU correspondent exactement aux drapeaux de l'architecture ARM. En cas de doute, se référer à \textit{\ref{sec:ISA}~\nameref{sec:ISA}} et \textit{\ref{subsec:Flags}~\nameref{subsec:Flags}} .

\begin{tabular}{|c|l|l|}
\hline
\textbf{Symbole} & \textbf{Nom} & \textbf{Description}\\
\hline

\texttt{N}	& \texttt{Negative}	& Résultat négatif\\
\hline
\texttt{Z}	& \texttt{Zero}		& Résultat nul\\
\hline
\texttt{C}	& \texttt{CarryOut}	& Retenue sortante (dépassement de capacité non-signé)\\
\hline
\texttt{V}	&  \texttt{Overflow}	& Dépassement de capacité signé\\

\hline
\end{tabular}
