\section{Présentation du projet}
\subsection{Le microprocesseur ARM Cortex-M0}
	La famille des ARM Cortex M regroupe des processeurs 32 bits. Ils peuvent être utilisés comme microprocesseur ou microcontrôleur. On les retrouve dans diverses applications : Arduino Due,  Machine à laver, distributeur de boissons...  Les cortex M vise en majorité le marché de l'embarqué.

	Le but de ce projet est de simuler le comportement d'un cortex M0 au moyen d'un logiciel de simulation électronique (logisim). L'idée est ici d'obtenir un système ayant un comportement similaire à un Cortex M0 et non une copie conforme du fait de la complexité d'un processeur réel.

\subsection{Le projet}
	Durant ce projet nous allons implémenter notre microprocesseur en le divisant en sous bloc à savoir :
\begin{itemize}
	\item Partie processeur
	\begin{itemize}
		\item ALU
		\item Banc de registres
		\item Contrôleur
	\end{itemize}
	\item Partie logicielle
	\begin{itemize}
		\item Assembleur
		\item Exportation sur FPGA
	\end{itemize}
\end{itemize}

\subsection{Logisim}

\paragraph{}
	Logisim est un programme permettant la modélisation et la simulation de circuit logique. La modélisation du circuit ne se fait que par dessin et glisser-déposer des différents éléments électroniques.
