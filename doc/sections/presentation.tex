\section{Présentation du projet}
\subsection{Le microprocesseur ARM Cortex-M0}
	La famille des ARM Cortex M regroupe des processeurs 32 bits. Ils peuvent être utilisés comme microprocesseur ou microcontrôleur. On les retrouve dans diverses applications : Arduino Due,  Machine à laver, distributeur de boissons...  Les cortex M vise en majorité le marché de l'embarqué.

	Le but de ce projet est de simuler le comportement d'un cortex M0 au moyen d'un logiciel de simulation électronique (logisim). L'idée est ici d'obtenir un système ayant un comportement similaire à un Cortex M0 et non une copie conforme du fait de la complexité d'un processeur réel.

\subsection{Assembleur}

	Le code binaire a exécuter est obtenu par l'assemblage d'instructions issus du jeu d'instructions ARMv7 (contre un jeu ARMv6 dans le cortex M0 réel).

\subsubsection{Syntaxe}
Syntaxe UAL\\
S: màj des drapeaux\\
<c>: condition\\
Rd: registre destination\\
<imm?>: immédiat\\
SP: registre de pointeur de pile en mémoire\\
opcode: code de l'instruction, peut occuper jusqu'à la taille indiquée\\
\[\]: argument optionnel\\

Example:
\begin{lstlisting}
.LBBH:                                
	ldr	r0, [sp, #4]
	ldr	r1, [sp, #28]
	cmp	r0, r1
	bge	.LBBK
	b	.LBBI
.LBBI:                                
	ldr	r0, [sp, #20]
	movs	r1, #1
	ands	r0, r1
	str	r0, [sp, #36]
	ldr	r0, [sp, #32]
	subs	r0, r1, r0
	str	r0, [sp, #32]
	ldr	r0, [sp, #32]
	ldr	r1, [sp, #36]
	lsls	r1, r1, #1
	adds	r0, r0, r1
	str	r0, [sp, #52]
	ldr	r0, [sp, #20]
	asrs	r0, r0, #1
	str	r0, [sp, #20]
	b	.LBBJ
\end{lstlisting}
